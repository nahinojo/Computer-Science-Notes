\documentclass[12pt]{article}
\usepackage{amsmath,amsfonts,amssymb,amsthm} % Math packages
\usepackage[mathscr]{euscript}
\usepackage[utf8]{inputenc}
\usepackage[T1]{fontenc}
\usepackage{titling}
\usepackage[mathscr]{euscript}
\let\euscr\mathscr \let\mathscr\relax% just so we can load this and rsfs
\usepackage[scr]{rsfso}
\usepackage{pifont}
\usepackage{geometry} % Adjust page margins
\usepackage{titlesec} % Adjust section and subsection formatting
\geometry{a4paper, left=1in, right=1in, top=1in, bottom=1in}

\title{
  The Solutions
  }
\author{Noah Hinojos}
\date{\today}

\titleformat*{\subsection}{\normalsize\bfseries}

\begin{document}
\pagenumbering{gobble}
\maketitle


\section*{Problem 1}
\subsection*{Question}
Divide. Check your answer by multiplication. \\
\begin{align*}
  \frac{x^2 -5x - 10}{x-8}
\end{align*}

\subsection*{Answer}

\begin{align*}
  &\frac{x^2 -5x - 10}{x-8} \\
  \\
  &\frac{(x-8)(x+3)+14}{x-8} \\
  \\
  &\frac{(x-8)(x+3)}{x-8} \ + \ \frac{14}{x-8} \\
  \\
  & x + 3 + \frac{14}{x-8} \\
\end{align*}
\newblock
\\
Now, you're probably wondering how I miraculously pulled the expression $(x-8)(x+3)+14$ out of my ass. Here's how can do this too:
\\ \\
Let's start by focusing on the numerator: $x^2 - 5x - 10$.
When you typically factor out these kinds of expressions, you're probably doing something like this in your head:
\\ \\
\begin{align*}
  x^2 - 5x -10 &= (x+a)(x+b) \\
\end{align*}
\newblock
Where, \\
\begin{align*}
  (a+b)x &= -5x \\
  ab &= -10
\end{align*}
Or, more simply:
\begin{align*}
  a+b &= -5 \\
  ab &= -10
\end{align*}
In other words, you're probably thinking 'What two numbers ($a$ and $b$) multiply to get -10 and add to get -5?'
But now you're at a dead end, there's no simple solution here.
\\ \\
Here's a better approach:
\begin{align*}
  x^2 - 5x -10 &= (x+a)(x+b)+c \\
\end{align*}
\newblock
Where,
\begin{align*}
  a+b &= -5 \\
  ab + c &= -10
\end{align*}
Compare this with the old approach. The only difference is that we have a bonus $+c$ to get -10. 
This makes things a lot easier.
\\ \\
Let's continue solving with this approach and you should see what I mean:
\begin{align*}
  x^2 - 5x -10 &= (x+a)(x+b)+c \\
\end{align*}
\newblock
Since we want to cancel out the $x-8$ in the denominator, we should set $a=-8$:
\begin{align*}
  x^2 - 5x -10 &= (x-8)(x+b)+c \\
\end{align*}
(Now I know that this doesn't \textit{fully} cancel out the $x-8$ in the denominator. 
But frankly, that $+c$ can be pulled out on its own later on).
\\ \\
Next, we still want $a$ and $b$ to summate to be $-5$. Or rather, we want to satisfy our equation from before:
\begin{align*}
  a+b = -5
\end{align*}
\newblock
Plugging in -8 for $a$, and solving:
\begin{align*}
  a+b &= -5 \\
  -8+b &= -5 \\
  b &= 3 \\
\end{align*}
Now that we've determined $a$ and $b$, let's solve for $c$ using the other equation from before:
\begin{align*}
  ab + c &= -10 \\
  (-8)(3) + c &= -10 \\
  -24 + c &= -10 \\
  c &= 14 \\
\end{align*}
Great! Let's put this all back into the equation:
\begin{align*}
  x^2 - 5x -10 &= (x+a)(x+b)+c \\
  x^2 - 5x -10 &= (x-8)(x+3)+14 \\
\end{align*}
And there you have it. You factored out the expression.
You can continue solving the division problem. 
\\ \\
Of course, you're going to end up with $\frac{14}{x-8}$ in your result. 
But I'd like to think the final answer is nonetheless better.
\section*{Problem 2}
\subsection*{Question}
Divide. \\
\begin{align*}
  \frac{2t^3-t+11}{t+3}
\end{align*}
\subsection*{Answer}
I'm going to take a slightly more advanced approach from Problem 1 and factor out the numerator like so:
\begin{align*}
  2t^3-t+11 &= (t+3)(at^2 + bt + c)+d \\
\end{align*}
We obviously want the numerator to have some factor of $(t+3)$ in it. 
But since the expression has a highest exponent of 3, we are expecting to factor out some high order expression.
Hence, we use $(at^2 + bt + c)$.
\\ \\
Let's continue:
\begin{align*}
  2t^3-t+11 &= (t+3)(at^2 + bt + c)+d \\
\end{align*}
This tells us that:
\begin{align*}
  3c + d &= 11 \\
  (3b + c)t &= -t \\
  (3a + b)t^2 &= 0t^2 \\
  at^3 &= - 2t^3 \\
\end{align*}
Or, more simply:
\begin{align*}
  3c + d &= 11 \\
  3b + c &= -1 \\
  3a + b &= 0 \\
  a &= - 2 \\
\end{align*}
Notice we have four equations with four unknowns. I'm not going to show the simple algebra but the result of these four equations gives us:
\begin{align*}
  a &= -2 \\
  b &= -6 \\
  c &= 17 \\
  d &= -40 \\
\end{align*}
Plugging it back in:
\begin{align*}
  2t^3-t+11 &= (t+3)(-2t^2 - 6t + 17)-40 \\
\end{align*}
Going back to the original problem:
\begin{align*}
  \frac{2t^3-t+11}{t+3} &= \frac{(t+3)(-2t^2 - 6t + 17)-40}{t+3} \\
  &= -2t^2 - 6t + 17 - \frac{40}{t+3} \\
\end{align*}
\end{document}