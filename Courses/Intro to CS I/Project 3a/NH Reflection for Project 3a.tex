\documentclass[12pt]{article}

\usepackage{amsmath,amsfonts,amssymb,amsthm} % Math packages
\usepackage[utf8]{inputenc}
\usepackage[T1]{fontenc}
\usepackage{makecell}
\usepackage{geometry} % Adjust page margins
\usepackage{titlesec} % Adjust section and subsection formatting
\geometry{a4paper, left=1in, right=1in, top=1in, bottom=1in}

\title{\textbf{Project Plan for Project 3a}}
\author{Noah Hinojos}
\date{\today}

\titleformat*{\subsection}{\normalsize\bfseries}

\begin{document}

\maketitle

\section*{Reflection}

The problem of developing an algorithm for finding the minimum and maximum integer I found somewhat challenging. 
\\ \\
I better understood how to simultaneously use for-loops and if-statements in tandem. Specificially, I used the if-statements within the main for-loop to reassign the values of $min\_int$ and $max\_int$.
\\ \\
Yet the main hiccup with my original design was that I misunderstood the typing of the user input. In my project plan, I incorrectly assumed that the numeric input float and not integer. Multiple of my tests mistakenly used float values as user inputs. Algorithmically, the numeric type wouldn't have changed much. It all really boiled down to what built-in function I wrapped the user input in, $float()$ or $int()$. Fortunately I was able to catch this mistake before finalizing the project.
\\ \\
Other than that, solving this problem was fairly straightforward. I didn't  use any external resources (such as Python's Documentation) to complete the project. I defintiely have to be more attentive to the prompt and its underlying assumptions in future projects in order not to mistakenly assume the wrong data type again.

\end{document}