\documentclass[12pt]{article}

\usepackage{amsmath,amsfonts,amssymb,amsthm} % Math packages
\usepackage[utf8]{inputenc}
\usepackage[T1]{fontenc}
\usepackage{titling}
\usepackage{pifont}
\usepackage{geometry} % Adjust page margins
\usepackage{titlesec} % Adjust section and subsection formatting
\geometry{a4paper, left=1in, right=1in, top=1in, bottom=1in}

\title{
  \textbf{CS-225: Discrete Structures in CS} \\
  Homework 3, Part 1
  }
\author{Noah Hinojos}
\date{\today}

\titleformat*{\subsection}{\normalsize\bfseries}

\begin{document}
\maketitle

\section*{Exercise Set 4.2}

\subsection*{Problem 28}
\underline{Suppositions}:
\\ \\
Suppose $n-m$ is even where $n$ and $m$ are integers.
\\ \\
\underline{Goal}:
\\ \\
Show that $n^3-m^3$ is even.
\\ \\
\underline{Deductions}:
\\ \\
Let $n-m = r$ where $r$ is an even integer.
\newblock
\\ \\
The product of an even integer and some other integer is even:
\begin{itemize}
  \item [$\centerdot$] Let $a$ be represent an even integer and $b$ be represent some other integer.
  \item [$\centerdot$] By defintion, $a = 2x$ where $x$ is some integer.
  \item [$\centerdot$] The product of $a$ and $b$ becomes $ab = 2xb$.
  \item [$\centerdot$] Let $z = xb$.
  \item [$\centerdot$] By closure, $z$ is an integer because it is the product of two integers.
  \item [$\centerdot$] $ab= 2xb = 2z$.
  \item [$\centerdot$] Let $y = ab$. 
  \item [$\centerdot$] $ab = y = 2z$
  \item [$\centerdot$] By defintion of even, $y$ is an even integer because $y = 2z$.
  \item [$\centerdot$] By equality, $ab$ is an even integer because $y$ is an even integer.
  \item [$\centerdot$] Hence the product of an even integer and some other integer is even.
\end{itemize}
\newblock
\\ \\
The resulting value of $n^3-m^3$ has a factor of $r$: 
\begin{align*}
  n^3-m^3 &= (n-m)(n^2+nm+m^2)\\
  &= r(n^2+nm+m^2)\\
\end{align*}
 $n^2+nm+m^2 = l$ where $l$ is some integer:
\begin{itemize}
  \item [$\centerdot$] Let $n^2+nm+m^2 = l$.
  \item [$\centerdot$] By closure, $m^2$, $mn$, and $n^2$ are all integers because each expression is the product of two integers.
  \item [$\centerdot$] Then also by closure, $l$ must be an integer because it is the sum of integers.
\end{itemize}
\newblock
\\
$n^3-m^3$ is an even integer:
\begin{itemize}
  \item [$\centerdot$] $n^3-m^3 = (n-m)(n^2+nm+m^2) = rl$.
  \item [$\centerdot$] $rl$ is even because it is the product of an even integer and some other integer, as shown before.
  \item [$\centerdot$] By equality, $n^3-m^3$ is even because $rl$ is even.
  \item [$\centerdot$] Hence, $n^3-m^3$ is an even integer.
\end{itemize}
\newblock
\\ \\
Thus, $n^3-m^3$ was shown to be even.
\\ \\
\underline{Conclusion}:
\\ \\
It was therefore shown that for all integers $n$ and $m$, if $n - m$ is even then $n^3 - m^3$ is even.
This was exemplified by how the even integer $n-m$ can be factored from the expression. 

\subsection*{Problem 36}
\underline{Suppositions}:
\\ \\
Let $n$ and $m$ be any two consecutive integers. 
\\ \\
\underline{Goal}:
\\ \\
Show that $n^2 - m^2$ is odd.
\\ \\
\underline{Deductions}:
\\ \\
The sum of an even integer and an odd integer is odd:
\begin{itemize}
  \item [$\centerdot$] Let $a$ be an even integer and $b$ be an odd integer.
  \item [$\centerdot$] By defintion, $a = 2c$ and $b = 2d+1$ where $c$ and $d$ are some integers.
  \item [$\centerdot$] The sum of $a$ and $b$ becomes $a+b = 2c+2d+1 = 2(c+d) + 1$.
  \item [$\centerdot$] Let $e = c + d$.
  \item [$\centerdot$] $e$ is an integer because it is the sum of two integers.
  \item [$\centerdot$] So, by substitution, $a+b = 2e+1$ where $e$ is some integer.
  \item [$\centerdot$] By the defintion of odd, $a+b$ is odd.
\end{itemize}
\newblock
\\ \\
The product of two odd integers is odd:
\begin{itemize}
  \item [$\centerdot$] Let $f$ be an odd integer and $g$ be an odd integer.
  \item [$\centerdot$] By defintion, $f = 2h+1$ and $g = 2i+1$ where $h$ and $i$ are some integers.
  \item [$\centerdot$] The product of $f$ and $g$ becomes $fg = (2h+1)(2i+1) = 4hi + 2h + 2i + 1$.
  \item [$\centerdot$] Let $j = 2hi$.
  \item [$\centerdot$] $j$ is an integer because it is the product of integers.
  \item [$\centerdot$] $fg = 4hi + 2h + 2i + 1 = 2j + 2h + 2i + 1 = 2(j+h+i)+1$
  \item [$\centerdot$] Let $k = h+i+j$.
  \item [$\centerdot$] $k$ is an integer because it is the sum of integers.
  \item [$\centerdot$] $ab = 2(h+i+j)+1 = 2k+1$
  \item [$\centerdot$] By the defintion of odd, $ab$ is odd.
\end{itemize}
\newblock
\\ \\
For the consecutive integers $n$ and $m$, one must be even and the other must be odd:
\begin{itemize}
  \item [$\centerdot$] Since the integers are consecutive, $n$ may be greater than or less than $m$. 
  $n$ may also be even or odd yet it, even so, in either case it cannot be assumed $m$ is even or odd.
  To prove that of $n$ and $m$ one is even and the other is odd, we must consider all cases when $n>m$ and $n<m$ as well as where $n$ is even or odd.
  This creates four total cases to exhaustively check:
  \item [$\centerdot$] In the case of \underline{$n>m$ and $n$ is even}:
  \begin{itemize}
    \item [$\centerdot$] Let $n = 2l$ where $l$ is some integer.
    \item [$\centerdot$] Then $m = n-1 = 2l-1 = 2l+1-2$
    \item [$\centerdot$] Let $o=2l+1$.
    \item [$\centerdot$] By defintion of odd, $o$ is odd.
    \item [$\centerdot$] Then $m = o - 2$
    \item [$\centerdot$] $m$ is odd because the difference between any odd integer and any even integer is odd (Thereom 4.2.1).
    \item [$\centerdot$] Hence $n$ is even and $m$ is odd.
  \end{itemize}
  \item [$\centerdot$] In the case of \underline{$n>m$ and $n$ is odd}:
  \begin{itemize}
    \item  Let $n = 2p+1$ where $p$ is some integer.
    \item  Then $m = 2n + 1 -1 = 2n$
    \item  By defintion of even, $m$ is even.
    \item  Hence $n$ is odd and $m$ is even.
  \end{itemize}
  \item [$\centerdot$] In the case of \underline{$n<m$ and $n$ is even}:
  \begin{itemize}
    \item  Let $n = 2r$ where $r$ is some integer.
    \item  Then $m = n+1 = 2r-1 = 2r+1-2$
    \item  Let $s=2l+1$.
    \item  By defintion of odd, $s$ is odd.
    \item  Then $m =s-2$
    \item  $m$ is odd because it is the difference between any odd integer and any even integer is odd (Thereom 4.2.1).
    \item  Hence $n$ is even and $m$ is odd.
  \end{itemize}
  \item [$\centerdot$] In the case of \underline{$n<m$ and $n$ is odd}:
  \begin{itemize}
    \item Let $n = 2t+1$ where $t$ is some integer.
    \item Then $m = n+1 = 2t+2 = 2(t+1)$
    \item Let $u=t+1$
    \item $u$ is an integer because it is the sum of two integers.
    \item By substitution, $m = 2(t+1) = 2u$
    \item By defintion of even, $m$ is even.
    \item Hence $n$ is odd and $m$ is even.
  \end{itemize}
  \item [$\centerdot$] Therefore, as shown in call four cases, for two consecutive integers one must be even and the other must be odd.
\end{itemize}
The expression can be factored out such that $n^2 - m^2 = (n+m)(n-m)$. 
\\ \\
\newblock
$n+m = v$ where $v$ is some odd integer:
\begin{itemize}
  \item [$\centerdot$] Let $n+m = v$.
  \item [$\centerdot$] By closure, $v$ is an integer because it is the sum two integers.
  \item [$\centerdot$] $v$ is an odd integer because it is the sum of an even integer and an odd integer is odd, as shown before.
\end{itemize} 
\newblock
$n-m = w$ where $w$ is some odd integer:
\begin{itemize}
  \item [$\centerdot$] Let $n-m = w$.
  \item [$\centerdot$] By closure, $w$ is an integer because it is the difference two integers.
  \item [$\centerdot$] $w$ is odd because it is the difference between any odd integer and any even integer is odd (Thereom 4.2.1).


\end{itemize} 
\newblock
$n^2 - m^2$ is an odd integer:
\begin{itemize}
  \item [$\centerdot$] $n^2 - m^2 = (n+m)(n-m) = vw$.
  \item [$\centerdot$] $vw$ is odd because it is the product of two odd integers, as shown before.
  \item [$\centerdot$] By equality, $n^2 - m^2$ is odd because $vw$ is odd.
\end{itemize}
\underline{Conclusion}:
\\ \\
Therefore, the difference of the squares of any two consecutive integers is odd. 
This was exemplified by how it can be rewritten as the product of two odd integers.
\section*{Exercise Set 4.3}
\subsection*{Problem 39}
This proof only holds true for $r= \frac{1}{4}$ and $s=\frac{1}{2}$. 
This does not mean it holds true for any $r$ and $s$. 
It is incorrect to prove be generalizing from a single example.
\section*{Exercise Set 4.4}
\subsection*{Problem 26}
\underline{Suppositions}:
\\ \\
Let $a$, $b$, and $c$ be integers. Suppose $ab|c$.
\\ \\
\underline{Goal}:
\\ \\
Prove $a|c$ and $b|c$.
\\ \\
\underline{Deductions}:
\\ \\
$c = abk$ for some integer $k$:
\begin{itemize}
  \item [$\centerdot$] Let $d = ab$. 
  \item [$\centerdot$] Due to closure, $d$ must be an integer because it is the product of two integers.
  \item [$\centerdot$] $ab|c = d|c$.
  \item [$\centerdot$] By definition of divisibility, $d$ is non-zero.
  \item [$\centerdot$] By definition of zero product property, $a$ and $b$ are also non-zero because $d=ab$.
  \item [$\centerdot$] By defintion of divisibility, $k$ is an integer such that $c = dk$.
  \item [$\centerdot$] $c = dk = abk$.
\end{itemize}
\newblock
\\ \\
$a|c$ must be true:
\begin{itemize}
  \item [$\centerdot$] Let $s = bk$.
  \item [$\centerdot$] Due to closure, $s$ must be an integer because it is the product of two integers.
  \item [$\centerdot$] $c = abk = as$.
  \item [$\centerdot$] By defintion of divisibility, $a|c$ because $c = sa$ where $s$ is an integer and $a$ is a non-zero integer.
\end{itemize}
\newblock
\\ \\
$b|c$ must be true:
\begin{itemize}
  \item [$\centerdot$] Let $t = ak$.
  \item [$\centerdot$] Due to closure, $t$ must be an integer because it is the product of two integers.
  \item [$\centerdot$] $c = abk = tb$.
  \item [$\centerdot$] By defintion of divisibility, $b|c$ because $c = tb$ where $t$ is an integer and $b$ is a non-zero integer.
\end{itemize}
\newblock
\\ \\
Thus, $a|c$ anmd $b|c$ was to be shown.
\\ \\
\underline{Conclusion}:
\\ \\
For all integers $a$, $b$, and $c$, if $ab|c$ then $a|c$ and $b|c$ must be true.
\section*{Canvas Problems}

\subsection*{1}
\underline{Suppositions}:
\\ \\
Let $s$ be rational.
\\ \\
\underline{Goal}:
\\ \\
Show that $9s^4 + \frac{3}{7}s -5$ is also rational. 
\\ \\
\underline{Deductions}:
\\ \\
By the definition of rational, $s = \frac{a}{b}$ where $a$ is some integer and $b$ is some non-zero integer.
\\ \\
By algrebra, the original equation can be rewritten such that:
\begin{align*}
  9s^4 + \frac{3}{7}s -5 &= \frac{9a^4}{b^4} + \frac{3a}{7b} - 5 \\
  &= \frac{63a^4}{7b^4} + \frac{3ab^3}{7b^4} - \frac{35b^4}{7b^4} \\
  &= \frac{63a^4 + 3ab^3 - 35b^4}{7b^4}
\end{align*}
\newblock
\\ \\
$63a^4 = e$ where $e$ is an integer:
\begin{itemize}
  \item [$\centerdot$] Let $c = a^2$. 
  \item [$\centerdot$] Due to closure, $c$ is an integer because it is the product of two integers.
  \item [$\centerdot$] Let $d = a^4$.
  \item [$\centerdot$] $d = a^4 = c^2$. 
  \item [$\centerdot$] Due to closure, $d$ is an integer because it is the product of two integers.
  \item [$\centerdot$] Let $e = 63a^4$.
  \item [$\centerdot$] $e = 63a^4 = 63d$.
  \item [$\centerdot$] Due to closure, $e$ is an integer because it is the product of two integers.
  \item [$\centerdot$] Hence $63a^4 = e$ where $e$ is an integer.
\end{itemize}
\newblock
\\ \\
$3ab^3 = h$ where $h$ is an integer:
\begin{itemize}
  \item [$\centerdot$] Let $f = b^2$.
  \item  [$\centerdot$] Due to zero product property and closure, $f$ is a non-zero integer because it is the product of two non-zero integers.
  \item [$\centerdot$] Let $g = b^3 = fb$.
  \item [$\centerdot$] Due to closure, $g$ is an integer because it is the product of two integers.
  \item [$\centerdot$] Let $h=3ab^3 = 3g$.
  \item [$\centerdot$] Due to closure, $h$ is an integer because it is the product of two integers.
  \item [$\centerdot$] Hence $3ab^3 = h$ where $h$ is an integer.
\end{itemize}
\newblock
\\ \\
$35b^4 = j$ where $j$ is an integer:
\begin{itemize}
  \item [$\centerdot$] Let $i = h^2$.
  \item [$\centerdot$] Due to closure, $i$ is an integer because it is the product of two integers.
  \item [$\centerdot$] Let $j = 35b^4 = 35i$.
  \item [$\centerdot$] Due to closure, $j$ is an integer because it is the product of two integers.
  \item [$\centerdot$] Hence $35b^4 = j$ where $j$ is an integer.
\end{itemize}
\newblock
\\ \\
$63a^4 + 3ab^3 - 35b^4=l$ where $l$ is an integer:
\begin{itemize}
  \item [$\centerdot$] $63a^4 + 3ab^3 - 35b^4 = e + f - j$.
  \item [$\centerdot$] Let $k = e + f$.
  \item [$\centerdot$] Due to closure, $k$ is an integer because it is the sum of two integers.
  \item [$\centerdot$] $63a^4 + 3ab^3 - 35b^4 = e + f - j = k - j$
  \item [$\centerdot$] Let $l = k-j$.
  \item [$\centerdot$] Due to closure, $l$ is an integer because it is the difference of two integers.
  \item [$\centerdot$] Hence $63a^4 + 3ab^3 - 35b^4 = l$ where $l$ is an integer.
\end{itemize}
\newblock
\\ \\
$7b^4 = n$ where $n$ is a non-zero integer:
\begin{itemize}
  \item [$\centerdot$] $7b^4 = 7f^2$.
  \item [$\centerdot$] Let $m = f^2$.
  \item [$\centerdot$] Due to zero product property and closure, $m$ is a non-zero integer because it is the product of two non-zero integers.
  \item [$\centerdot$] $7b^4 = 7f^2 = 7m$.
  \item [$\centerdot$] Let $n = 7m$.
  \item [$\centerdot$] Due to zero product property and closure, $n$ is a non-zero integer because it is the product of two non-zero integers.
  \item [$\centerdot$] Hence $7b^4 = n$ where $n$ is a noon-zero integer.
\end{itemize}
\newblock
\\ \\
$9s^4 + \frac{3}{7}s -5$ is rational:
\begin{itemize}
  \item [$\centerdot$] $\frac{63a^4 + 3ab^3 - 35b^4}{7b^4} = \frac{l}{n}$.
  \item [$\centerdot$] By definition of rational, $\frac{63a^4 + 3ab^3 - 35b^4}{7b^4}$ is rational
  because it can be expressed as $\frac{l}{n}$ where $l$ is an integer and $n$ is a non-zero integer.
  \item [$\centerdot$] Recall $9s^4 + \frac{3}{7}s -5 = \frac{63a^4 + 3ab^3 - 35b^4}{7b^4}$.
  \item [$\centerdot$] By equality, $9s^4 + \frac{3}{7}s -5$ is rational because $\frac{63a^4 + 3ab^3 - 35b^4}{7b^4}$ is rational.
  \item [$\centerdot$] Hence $9s^4 + \frac{3}{7}s -5$ is rational.
\end{itemize}
\newblock
\\ \\
Thus $9s^4 + \frac{3}{7}s -5$ was shown to be rational. \\ \\
\underline{Conclusion}:
\\ \\
It must must be true that $9s^4 + \frac{3}{7}s -5$ is rational. 
This was exmplified by how, through algebra, it can be represeted as an integer divided by a non-zero integer.

\subsection*{2}
\underline{Suppositions}:
\\ \\
Suppose $a|b$ and $a|(b^2 - c)$ where $a$, $b$, and $c$ are integers.
\\ \\
\underline{Goal}:
\\ \\
Prove $a|c$ must be true.
\\ \\
\underline{Deductions}:
\\ \\
$\frac{c}{a} = am^2 - n$ where $m$ and $n$ are non-zero integers:
\begin{itemize}
  \item  [$\centerdot$] By definition of divisibility, $a$ is a non-zero integer because $a|b$.
  \item  [$\centerdot$] By defintion of divisibility, $a|b$ implies $b=am$ for some non-zero integer $m$.
  \item  [$\centerdot$] By defintion of divisibility, $a|(b^2-c)$ states $b^2-c=an$ for some non-zero integer $n$.
  \item  [$\centerdot$] By substitution, $b^2-c=an$ can be rewritten:
  \begin{align*}
    b^2-c&=an\\ 
    (am)^2-c&=an \\
    a^2m^2-c&=an \\
    c &= a^2m^2 - an\\\section*{Example} 
    % Example usage of the custom function
    Here is a list using a custom \texttt{\textbackslash centerdot} marker:
    \myitemlist{
        \item First item
        \item Second item
        \item Third item
    }
     Due to closure, $o$ is an integer because it is the product of two integers.
  \item [$\centerdot$] $\frac{c}{a} = am^2 - n = ao - n$
  \item [$\centerdot$] Let $p=ao$.
  \item [$\centerdot$] Due to closure, $p$ is an integer because it is the product of two integers.
  \item [$\centerdot$] $\frac{c}{a} = ao - n = p - n$
  \item [$\centerdot$] Let $q=p-n$.
  \item [$\centerdot$] Due to closure, $q$ is an integer because it is the difference of two integers.
  \item [$\centerdot$] $\frac{c}{a} =p-n=q$
  \item [$\centerdot$] Hence $\frac{c}{a} = q$ where $q$ is an integer.
\end{itemize}
\newblock
\\ \\
$a|c$ must be true.
\begin{itemize}
  \item [$\centerdot$] Recall $\frac{c}{a}=q$ where $c$ and $q$ are integers, and $a$ is a non-zero integer.
  \item [$\centerdot$] $\frac{c}{a}=q \Rightarrow c=aq$
  \item [$\centerdot$] By defintion of divisibility, $a|c$ because $c=aq$ where $q$ is an integer and $a$ is a non-zero integer.
\end{itemize}
\underline{Conclusion}:
\\ \\
It is therefore true that $a|c$. 
This was exemplified by how $a$ is definitively a non zero integer,
and by how $c$ is an integer through algrebra.
\end{document}