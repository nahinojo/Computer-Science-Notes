\documentclass[12pt]{article}

\usepackage{amsmath,amsfonts,amssymb,amsthm} % Math packages
\usepackage[utf8]{inputenc}
\usepackage[T1]{fontenc}
\usepackage{titling}
\usepackage{pifont}
\usepackage{geometry} % Adjust page margins
\usepackage{titlesec} % Adjust section and subsection formatting
\geometry{a4paper, left=1in, right=1in, top=1in, bottom=1in}

\title{
  \textbf{CS-225: Discrete Structures in CS} \\
  Homework 3, Part 2
  }
\author{Noah Hinojos}
\date{\today}

\titleformat*{\subsection}{\normalsize\bfseries}

\begin{document}
\maketitle

\section*{Exercise Set 4.7}
\subsection*{18}
\underline{Suppositions}:
Suppose not. Let $a$ be a rational number, $b$ be a non-zero rational number,
$r$ be an irrational number, and $a+br$ be an rational number.
\\ \\
\underline{Goal}:
\\ \\
We must arrive at a contradiction.
\\ \\
\underline{Deductions}:
\\ \\
$r= \frac{fgd - fch}{edh}$ where $c$ and $g$ are integers; and $d$, $e$, $f$, and $h$ are non-zero integers.:
\begin{itemize}
  \item [$\centerdot$] By definition of rational, $a = \frac{c}{d}$ where $c$ and $d$ are integers and $d$ is non-zero. 
  \item [$\centerdot$] By definition of rational, $b = \frac{e}{f}$ where $e$ and $f$ are integers and $f$ is non-zero.
  \item [$\centerdot$] By eqaulity, $\frac{e}{f}$ is non-zero because $b$ is non-zero.
  \item [$\centerdot$] By algebra, $e$ is non-zero because $\frac{e}{f}$ is non-zero.
  \item [$\centerdot$] By definition of rational, $a+br = \frac{g}{h}$ where $g,h$ are integers and $h \neq 0$.
  \item [$\centerdot$] By substitution and algebra, $r$ can be represented as follows:
  \begin{align*}
    a+br &= \frac{g}{h} \\
    \left(\frac{c}{d}\right) + \left(\frac{e}{f}\right)r &= \frac{g}{h}\\
    \frac{c}{d} + \left(\frac{e}{f}\right)r &= \frac{g}{h}\\
    \left(\frac{e}{f}\right)r &= \frac{g}{h} - \frac{c}{d} \\
    r &= \left(\frac{f}{e}\right)\left(\frac{g}{h} - \frac{c}{d}\right)  \\
    &= \frac{fg}{eh} - \frac{fc}{ed} \\
    &= \frac{fgd}{ehd} - \frac{fch}{edh} \\
    &= \frac{fgd - fch}{edh} \\
  \end{align*}
  \item [$\centerdot$] Hence $r = \frac{fgd - fch}{edh}$.
\end{itemize}
\newblock
\\ \\
The product of three integers is an integer:
\begin{itemize}
  \item [$\centerdot$] Let $x$, $y$, and $z$ be some integers.
  \item [$\centerdot$] Then their product is $xyx$.
  \item [$\centerdot$] Let $w = xy$.
  \item [$\centerdot$] Due to closure, $w$ is an integer because it is the product of two integers.
  \item [$\centerdot$] Let $v = xyz$.
  \item [$\centerdot$] $v = xyz = wz$.
  \item [$\centerdot$] Due to closure, $v$ is an integer because it is the product of two integers.
  \item [$\centerdot$] Since $xyz = v$, $xyz$ is also an integer.
  \item [$\centerdot$] Hence, the product of three integers is an integer.
\end{itemize}
\newblock
\\ \\
The product of three non-zero integers is a non-zero integer:
\begin{itemize}
  \item [$\centerdot$] Let $i$, $t$, and $u$ be some non-zero integers.
  \item [$\centerdot$] Then their product is $stu$.
  \item [$\centerdot$] Let $p = st$.
  \item [$\centerdot$] Due to closure and zero product property, $p$ is an non-zero integer because it is the product of two non-zero integers.
  \item [$\centerdot$] Let $q = stu$.
  \item [$\centerdot$] $q = stu = pu$.
  \item [$\centerdot$] Due to closure and zero product property, $q$ is an non-zero integer because it is the product of two non-zero integers.
  \item [$\centerdot$] Since $stu = q$, $stu$ is also an non-zero integer.
  \item [$\centerdot$] Hence, the product of three non-zero integers is a non-zero integer.
\end{itemize}
\newblock
\\ \\
$r$ is rational:
\begin{itemize}
  \item [$\centerdot$] Recall $r = \frac{fgd - fch}{edh}$.
  \item [$\centerdot$] Let $i = fgd$.
  \item [$\centerdot$] $i$ is an integer because it is the product of three integers.
  \item [$\centerdot$] Let $j = fch$.
  \item [$\centerdot$] $j$ is an integer because it is the product of three integers.
  \item [$\centerdot$] Let $k = edh$.
  \item [$\centerdot$] $k$ is a non-zero integer because it is the product of three non-zero integers.
  \item [$\centerdot$] $r = \frac{fgd - fch}{edh} = \frac{i - j}{k}$.
  \item [$\centerdot$] Let $l = i - j$. 
  \item [$\centerdot$] Due to closure, $l$ is an integer because it is the difference of two integers.
  \item [$\centerdot$] $r = \frac{i - j}{k} = \frac{l}{k}$.
  \item [$\centerdot$] By defintion of rational, $r$ is a rational number 
  because it can be expressed as $\frac{l}{k}$ where $l$ is an integer and $k$ is a non-zero integer.
\end{itemize}
\newblock
\\ \\
We arrived at a contradiction because $r$ is rational. This contradicts the supposition that $r$ is irrational.
\\ \\
\underline{Conclusion}:
\\ \\
Therefore, it must be true that if $a$ and $b$ are rational numbers, $b\neq0$, and $r$ is an irrational number, then $a+br$ is irraitonal.


\subsection*{27}
\underline{Suppositions}:
\\ \\
Suppose not. Let's assume $r$ and $s$ are some positive real numbers and $\sqrt{r+s} = \sqrt{r} + \sqrt{s}$
\\ \\
\underline{Goal}:
\\ \\
Arrive at contradiction.
\\ \\
\underline{Deductions}:
\\ \\
For any positive real number $a$, $\sqrt{a}$ is a non-zero real number:
\begin{itemize}
  \item [$\centerdot$] Let $\sqrt{a} = b$
  \item [$\centerdot$] By algebra, $\sqrt{a} = b \Rightarrow b^2 = a$
  \item [$\centerdot$] By equality, $b^2$ is a positive real number because $a$ is a positive real number.
  \item [$\centerdot$] By zero product property, $b$ is a non zero real number because $b^2$ is a non-zero real number.
  \item [$\centerdot$] By equality, $\sqrt{a}$ is a non-zero real number because $b$ is a non-zero real number.
  \item [$\centerdot$] Hence, $\sqrt{a}$ is a non-zero real number.
\end{itemize}
\newblock
\\ \\
The product of any three non-zero real numbers is a non-zero real number:
\begin{itemize}
  \item [$\centerdot$] Let $x$, $y$, and $z$ be non-zero real numbers.
  \item [$\centerdot$] Then the
  \\ \\ = wz$.
  \item [$\centerdot$] Due to zero product property, $v$ is a non-zero real number.
  \item [$\centerdot$] By equality, $xyz$ is also a non-zero real number because $v$ is a non-zero real number.
  \item [$\centerdot$] Hence, the product of three non-zero real numbers $xyz$ is a non-zero real number.
\end{itemize}
\newblock
\\ \\
Through algebra, $\sqrt{r+s} = \sqrt{r} + \sqrt{s} \Rightarrow 0 = 2\sqrt{r}\sqrt{s}$:
\begin{align*}
  \sqrt{r+s} &= \sqrt{r} + \sqrt{s} \\
  r+s &= (\sqrt{r} + \sqrt{s})^2 \\
  r+s&= r + 2\sqrt{r}\sqrt{s} + s\\
  0 &= 2\sqrt{r}\sqrt{s}\\
\end{align*}
\newblock
\\ \\
If $r$ and $s$ are positive real numbers, then $2\sqrt{r}\sqrt{s} \neq 0$.
\begin{itemize}
  \item [$\centerdot$] $\sqrt{r}$ is a non-zero real number since $r$ is a positive real number, as shown before.
  \item [$\centerdot$] $\sqrt{s}$ is a non-zero real number since $s$ is a positive real number, as shown before.
  \item [$\centerdot$] $2\sqrt{r}\sqrt{s}$ must be a non-zero real number because 
  it is the product of three non-zero real numbers, as shown before.
  \item [$\centerdot$] Hence $2\sqrt{r}\sqrt{s} \neq 0$.
\end{itemize}
\newblock
\\ \\
The supposition is contradictory because the supposed equation requires $0 = 2\sqrt{r}\sqrt{s}$ to also be true. 
However, $2\sqrt{r}\sqrt{s} \neq 0$ because $r$ and $s$ are positive real integers.
\newblock
\\ \\
\underline{Conclusion}:
\\ \\
Therefore, for all positive real numbers $r$ and $s$, $\sqrt{r+s} \neq \sqrt{r} + \sqrt{s}$ must be true.
\subsection*{28}
\underline{Suppositions}:
\\ \\
Suppose not. Assume $a|b$, $a \nmid c$, and $a |(b+c)$.
\\ \\
\underline{Goal}:
\\ \\
Arrive at a contradiction.
\\ \\
\underline{Deductions}:
\\ \\
$a|c$ must be true.
\begin{itemize}
  \item [$\centerdot$] $a|b$ implies $b = ap$ where $b$ and $p$ are integers and $a$ is a non-zero integer.
  \item [$\centerdot$] By zero product property, $a$ and $p$ are non-zero because $b$ is non-zero.
  \item [$\centerdot$] $a|(b+c)$ implies $b+c = aq$ where $q$ is some integer.
  \item [$\centerdot$] By algebra, $c = aq -b$
  \item [$\centerdot$] By substitution, $c = aq - ap$
  \item [$\centerdot$] By algebra, $c = a(q-p)$
  \item [$\centerdot$] Let $q-p = r$
  \item [$\centerdot$] By closure, $r$ is an integer because it is the difference of two integers.
  \item [$\centerdot$] $c = a(q-p) = ar$
  \item [$\centerdot$] By closure, $c$ is an integer because it is the product of two integers.
  \item [$\centerdot$] By defintion of divisibility, $a|c$ because $c$ is an integer and $a$ is a non-zero integer.
\end{itemize}
\newblock
\\ \\
It was therefore shown that $a | c$. This contradicts the supposition that $a \nmid c$.
\\ \\
\underline{Conclusion}:
\\ \\
For all integers $a$, $b$, $c$, if $a|b$ and $a \nmid c$, then $a \nmid (b+c)$.
\section*{4.8}
\subsection*{18 - a}
\underline{Suppositions}:
\\ \\
Suppose $a$ is an odd integer.
\\ \\
\underline{Goal}:
\\ \\
Prove $a^3$ is odd.
\\ \\
\underline{Deductions}:
\\ \\
The product of two odd integers is odd:
\begin{itemize}
  \item [$\centerdot$] Let the product of two odd integers be $xy$, where $x$ and $y$ are odd integers.
  \item [$\centerdot$] By definition of odd, $x=2s+1$ and $y=2t+1$ where $s$ and $t$ are integers.
  \item [$\centerdot$] By algebra, $xy$ can be rewritten:
  \begin{align*}
    xy &= (2s+1)(2t+1) \\
    &= 4st+2s+2t+1 \\
    &= 2(2st+s+t)+1
  \end{align*}
  \item [$\centerdot$] Let $n = st$.
  \item [$\centerdot$] By closure, $n$ is an integer because it is the product of two integers.
  \item [$\centerdot$] Let $w = 2st$.
  \item [$\centerdot$] By algebra, $w = 2st = 2n$.
  \item [$\centerdot$] By closure, $w$ is an integer because it is the product of two integers.
  \item [$\centerdot$] Let $z = s + t$.
  \item [$\centerdot$] By closure, $z$ is an integer because it is the sum of two integers.
  \item [$\centerdot$] Let $v = st+s+t$.
  \item [$\centerdot$] By algebra, $v = st+s+t= w+z$.
  \item [$\centerdot$] By closure, $v$ is an integer because it is the sum of two integers.
  \item [$\centerdot$] Let $t=xy$.
  \item [$\centerdot$] By algebra and substitution, $t = xy = 2(st+s+t)+1 = 2v + 1$.
  \item [$\centerdot$] By defintion of odd, $t$ is an odd integer.
  \item [$\centerdot$] By equality, $xy$ is an odd integer since $t$ is an odd integer.
  \item [$\centerdot$] Therefore, the product of two odd integers is an odd integer.
\end{itemize}
$a^3$ is an odd integer:
\begin{itemize}
  \item [$\centerdot$] Let $b = a^2$.
  \item [$\centerdot$] Since $b$ is the product of two odd integers, $b$ is an odd integer, as shown before.
  \item [$\centerdot$] Let $c = ab$.
  \item [$\centerdot$] Since $c$ is the product of two odd integers, $c$ is an odd integer, as shown before.
  \item [$\centerdot$] By algebra, $c = ab = a(a^2) = a^3$
  \item [$\centerdot$] By equality, $a^3$ is an odd integer because $c$ is an odd integer.
\end{itemize}
\newblock
\\ \\
Thus, if $a$ is an odd integer then $a^3$ is an odd integer.
\\ \\
\underline{Conclusion}:
\\ \\
For every integer $a$, if $a^3$ is even then $a$ is even.
\subsection*{18 - b}
\underline{Suppositions}:
\\ \\
Suppose not. Let's assume $\sqrt[3]{2} = p$ and $p$ is a rational number.
\\ \\ 
\underline{Goal}:
\\ \\
Arrive at a contradiction.
\\ \\
\underline{Deductions}:
\\ \\
$p$ is irrational:
\begin{itemize}
  \item [$\centerdot$] By definition of rational, $p = \frac{a}{b}$ where $a$ is an integer and $b$ is a non-zero integer.
  \item [$\centerdot$] This also implies $a$ and $b$ also have no common factors.
  \item [$\centerdot$] By algebra,
  \begin{align*}
    p =\sqrt[3]{2} &=\frac{a}{b} \\
    \sqrt[3]{2} &= \frac{a}{b} \\
    2 &= \frac{a^3}{b^3} \\
    2b^3 &= a^3\\
  \end{align*}
  \item [$\centerdot$] Let $n = b^3$. 
  \item [$\centerdot$] $n$ is an integer because it is the product of three integers (See Exercise Set 4.7 \#18  above).
  \item [$\centerdot$] Let $m = 2b^3$.
  \item [$\centerdot$] By algebra, $m = 2b^3 = 2n$.
  \item [$\centerdot$] By definition of even, $m$ is an even integer.
  \item [$\centerdot$] By algebra, $a^3 = 2b^3 = m$.
  \item [$\centerdot$] By definition of even, $a^3$ is an even integer. Therefore, $a$ must also be even.
  \item [$\centerdot$] Then $a = 2n$ where $n$ is an integer.
  \item [$\centerdot$] By algebra, $2b^3 = a^3 = (2n)^3 = 2\cdot2^2\cdot n^3 = 2(4n^3)$ 
  \item [$\centerdot$] Let $o = n^3$.
  \item [$\centerdot$] $o$ is an integer because it is the product of three integers (See Exercise Set 4.7 \#18  above).
  \item [$\centerdot$] Let $k = 2n^3$.
  \item [$\centerdot$] By algebra, $k = 2n^3 = 2o$.
  \item [$\centerdot$] By definition of even, $k$ is an even integer.
  \item [$\centerdot$] Let $j = 4n^3$.
  \item [$\centerdot$] By algebra, $j = 4n^3 = 2k$.
  \item [$\centerdot$] By definition of even, $j$ is an even integer.
  \item [$\centerdot$] By algebra, $2b^3 = 2(4n^3) = 2j$ 
  \item [$\centerdot$] By algebra, $b^3 = j$
  \item [$\centerdot$] By definition of even, $b^3$ is an even integer. Therefore, $b$ must also be even.
  \item [$\centerdot$] Since $a$ and $b$ are even, they have a common factor of 2.
  \item [$\centerdot$] However, this contradicts the prior deduction that $a$ and $b$ must have no common factors.
  \item [$\centerdot$] Hence, $p$ must be irrational.
\end{itemize}
\newblock
\\ \\
Since $p$ is irrational, it contradicts the supposition that $p$ is rational.
\\ \\
\underline{Conclusion}:
\\ \\
$\sqrt[3]{2}$ is irrational.
\end{document}

