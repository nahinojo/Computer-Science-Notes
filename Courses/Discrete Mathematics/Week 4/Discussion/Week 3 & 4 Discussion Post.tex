\documentclass[12pt]{article}
\usepackage{amsmath,amsfonts,amssymb,amsthm} % Math packages
\usepackage[mathscr]{euscript}
\usepackage[utf8]{inputenc}
\usepackage[T1]{fontenc}
\usepackage{titling}
\usepackage[mathscr]{euscript}
\let\euscr\mathscr \let\mathscr\relax% just so we can load this and rsfs
\usepackage[scr]{rsfso}
\usepackage{pifont}
\usepackage{geometry} % Adjust page margins
\usepackage{titlesec} % Adjust section and subsection formatting
\geometry{a4paper, left=1in, right=1in, top=1in, bottom=1in}

\newcommand{\xlist}[1]{
    \begin{itemize}
        \renewcommand{\labelitemi}{$\centerdot$}
        #1
    \end{itemize}
    \newblock
    \\ \\
}

\newcommand{\xsupposition}[1]{
    \underline{Suppositions}:
    \\ \\
    #1
    \\ \\
}

\newcommand{\xgoal}[1]{
    \underline{Goal}:
    \\ \\
    #1
    \\ \\
}

\newcommand{\xdeductions}{
    \underline{Deductions}:
    \\ \\
}

\newcommand{\xconclusion}[1]{
    \underline{Conclusion}:
    \\ \\
    #1
    \\ \\
}

\title{
  \textbf{CS-225: Discrete Structures in CS} \\
  Initial- Post Wk 3 \& 4 Problem Sets \#7 and \#11
  }
\author{Noah Hinojos}
\date{\today}

\titleformat*{\subsection}{\normalsize\bfseries}

\begin{document}
\pagenumbering{gobble}
\maketitle

\section*{7}
\subsection*{Question}
Prove the following statement (using either direct or indirect proof method): \\ \\
For every real number $x$, if $x^5 - 4x^4 + 3x^3 - x^2 + 3x - 4 \geq 0$, then $x \geq 0$.

\subsection*{Answer}
[PROOF BY CONTRAPOSITIVE] \\ \\
\xsupposition{$x$ is a real number and $x < 0$.}
\xgoal{Prove that $x^5 - 4x^4 + 3x^3 - x^2 + 3x - 4 < 0$.}
\xdeductions
(i) $r^p > 0$ where $r$ is some non-zero real number and $p$ is some even integer.
\xlist{
  \item Let $r$ be some non-zero real number.
  \item Let $p$ be an even integer.
  \item Since $p$ is even, it can be represented as $p=2n$ where $n$ is some integer.
  \item $r^p = r^{2n} = (r^n)^2$.
  \item $(r^n)^2 > 0$ because the square of any non-zero number is positive.
  \item By equality, $r^p > 0$ because $(r^n)^2 > 0$.
  \item Hence, $r^p > 0$ where $r$ is some real number and $p$ is some even integer.
}
(ii) $s^q < 0$ where $s$ is some negative real number and $q$ is some odd integer.
\xlist{
    \item Let $s$ be some negative real number.
    \item Let $q$ be some odd integer.
    \item Since $q$ is odd, it can be represented as $q=2m + 1$ where $m$ is some integer.
    \item $s^q = s^{2m + 1} = (s^m)^2$.
    \item $(s^m)^2 > 0$ because the square of a non-zero real number is positive.
    \item $s(s^m)^2 < 0$ because it is the product of some positive real number $(s^m)^2$ and some negative real number $s$.
    \item By equality, $s^q < 0$ because $s(s^m)^2 < 0$.
}
(iii) $3x < 0$
\xlist{
  \item The product of a positive real number and a negative real number is a negative real number.
}
(iv) $-x^2 < 0$
\xlist{
  \item $x^2 > 0$ because the exponent is an even integer. See Deduction (i).
  \item Then $-x^2 < 0$ because it is the opposite sign of a positive number.
}
(v) $3x^3 < 0$
\xlist{
  \item $x^3 < 0$ because the exponent is a negative integer. See Deduction (ii).
  \item Then $3x^3 < 0$ because it is the product of a positive and negative number.
  
}
(vi) $-4x^4 < 0$
\xlist{
  \item $x^4 > 0$ because the exponent is an even integer. See Deduction (i).
  \item Then $-4x^4 < 0$ because it is the product of a positive and negative number.
}
(vii) $x^5 < 0$
\xlist{
  \item $x^5 < 0$ because the exponent is a negative integer. See Deduction (ii).
}
By Deductions (iii) through (vii), every individual expression in $x^5 - 4x^4 + 3x^3 - x^2 + 3x - 4$ is less than 0. 
Then by closure, $x^5 - 4x^4 + 3x^3 - x^2 + 3x - 4 < 0$. \\ \\
\xconclusion{$x^5 - 4x^4 + 3x^3 - x^2 + 3x - 4 < 0$.}

\section*{11}
\subsection*{Question}

a) Find $\bigcup_{i=1}^{4}A_i$ and $\bigcap_{i=1}^{4}A_i$ if for every positivie integer $i$.
\begin{itemize}
  \item [I.] $A_i = \{i, i+1, i+2, ...\}$. Are $A_1, A_2, A_3, A_4$ mutually disjoint? Justify your answer.
  \item [II.] $A_i = \{0, i\}$. Are $A_1, A_2, A_3, A_4$ mutually disjoint? Justify your answer.
  \item [III.] $A_i = (0,i)$, that is, the set of real numbers $x$ with $0<x<i$. Are $A_1, A_2, A_3, A_4$ mutually disjoint? Justify your answer.
  \item [IV.] $A_i = (0,\infty)$, that is, the set of real numbers $x$ with $x>i$. Are $A_1, A_2, A_3, A_4$ mutually disjoint? Justify your answer.
\end{itemize}
\newblock
\\ \\
b) Let $R$, $S$, and $T$, be defined as follows:
\begin{itemize}
  \item [] $R = \{x \in Z | x \text{ is divisible by } 2\}$ \\
  \item [] $S = \{x \in Z | x \text{ is divisible by } 6\}$ \\
  \item [] $T = \{x \in Z | x \text{ is divisible by } 5\}$
\end{itemize}
\newblock
\\
Prove or disprove each of the following statements.
\begin{itemize}
  \item [I.] $R \subseteq T$
  \item [II.] $T \subseteq R$
  \item [III.] $T \subseteq S$
\end{itemize}

\subsection*{Answer}
\textbf{a - I)}
\begin{align*}
  \bigcup_{i=1}^{4}A_i &= \bigcup_{i=1}^{4}\{i, i+1, i+2, ...\} \\
  &= \{1, 1+1, 1+2, ...\} \cup \{2, 2+1, 2+2, ...\} \cup \{3, 3+1, 3+2, ...\} \cup \{4, 4+1, 4+2, ...\} \\
  &= \{1, 2, 3, ... \} \cup \{2, 3, 4, ... \} \cup \{3, 4, 5, ... \} \cup \{4, 5, 6, ... \} \\
  &= \{1, 2, 3, 4, 5, 6, ... \}
\end{align*}
\begin{align*}
  \bigcap_{i=1}^{4}A_i &= \bigcap_{i=1}^{4}\{i, i+1, i+2, ...\} \\
  &= \{1, 1+1, 1+2, ...\} \cap \{2, 2+1, 2+2, ...\} \cap \{3, 3+1, 3+2, ...\} \cap \{4, 4+1, 4+2, ...\} \\
  &= \{1, 2, 3, ... \} \cap \{2, 3, 4, ... \} \cap \{3, 4, 5, ... \} \cap \{4, 5, 6, ... \} \\
  &= \{4, 5, 6, ... \}
\end{align*}
No, the sets $A_1$, $A_2$, $A_3$, and $A_4$ are not mutually disjoint. This is because they all share the elements all integers that are greater than or equal to 4.
\\ \\
\textbf{a - II)}
\begin{align*}
  \bigcup_{i=1}^{4}A_i &= \bigcup_{i=1}^{4}\{0, i\} \\
  &= \{0, 1\} \cup \{0, 2\} \cup \{0, 3\} \cup \{0, 4\} \\
  &= \{0, 1, 2, 3, 4\}
\end{align*}
\begin{align*}
  \bigcap_{i=1}^{4}A_i &= \bigcap_{i=1}^{4}\{0, i\} \\
  &= \{0, 1\} \cap \{0, 2\} \cap \{0, 3\} \cap \{0, 4\} \\
  &= \{0\}
\end{align*}
No, the sets $A_1$, $A_2$, $A_3$, and $A_4$ are not mutually disjoint. This is because they all share the element 0.
\\ \\
\textbf{a - III)}
\begin{align*}
  \bigcup_{i=1}^{4}A_i &= \bigcup_{i=1}^{4}(0, i) \\
  &= (0, 1) \cup (0, 2) \cup (0, 3) \cup (0, 4) \\
  &= (0, 4)
\end{align*}
\begin{align*}
  \bigcap_{i=1}^{4}A_i &= \bigcap_{i=1}^{4}(0, i) \\
  &= (0, 1) \cap (0, 2) \cap (0, 3) \cap (0, 4) \\
  &= (0, 1)
\end{align*}
No, the sets $A_1$, $A_2$, $A_3$, and $A_4$ are not mutually disjoint. This is because they all share the set of numbers $(0, 1)$.
\\ \\
\textbf{a - IV)}
\begin{align*}
  \bigcup_{i=1}^{4}A_i &= \bigcup_{i=1}^{4}(i, \infty) \\
  &= (1, \infty) \cup (2, \infty) \cup (3, \infty) \cup (4, \infty) \\
  &= (1, \infty)
\end{align*}
\begin{align*}
  \bigcap_{i=1}^{4}A_i &= \bigcap_{i=1}^{4}(i, \infty) \\
  &= (1, \infty) \cap (2, \infty) \cap (3, \infty) \cap (4, \infty) \\
  &= (4, \infty)
\end{align*}
No, the sets $A_1$, $A_2$, $A_3$, and $A_4$ are not mutually disjoint. This is because they all share the set of numbers $(4, \infty)$.
\\ \\
\textbf{b - I)} \\ \\
\newblock
[DISPROOF BY COUNTEREXAMPLE]
\\ \\
\xsupposition{None.}
\xgoal{Disprove $R \subseteq T$.}
\xdeductions
$4 \in R$ but $4 \not\in T$
\xlist{
  \item $4 \in R$ because $4$ is divisible by 2.
  \item $4 \not\in T$ because $4$ is not divisible by 8.
  \item Hence, $4 \in R$ but $4 \not\in T$.
}
By definition of subset, $R \not\subseteq T$ since there exists an element in $R$ that is not in $T$.
\\ \\
\xconclusion{$R \not\subseteq T$}
\\ \\
\textbf{b - II)} \\ \\
\newblock
[ELEMENTAL PROOF]
\\ \\
\xsupposition{Let $x$ be a particular but arbitrarily chosen element of $T$.}
\xgoal{Prove $T \subseteq R$.}
\xdeductions
$x \in R$
\xlist{
  \item By definition of divisibility, $x = 8n$ where $n$ is some integer. This is because $x$ is divisible by 8.
  \item Let $m = 4n$.
  \item By closure, $m$ is some integer because it is the product of two integers.
  \item Then through substitution, $x = 8n = 2m$.
  \item By definition of divisibility, $x$ is divisible by 2 because $x = 2m$.
  \item Therefore, $x \in R$.
}
By definition of subset, $T \subseteq R$ because the arbitrarily but particular element $x$ of $T$ is in $R$. 
\\ \\
\xconclusion{$T \subseteq R$}
\textbf{b - III)} \\ \\
\newblock
[ELEMENTAL PROOF]
\\ \\
\xsupposition{None.}
\xgoal{Disprove $S \subseteq T$.}
\xdeductions
$12 \in S$ but $12 \not\in T$
\xlist{
  \item $12 \in S$ because $12$ is divisible by 6.
  \item $12 \not\in T$ because $12$ is not divisible by 8.
  \item Hence, $12 \in S$ but $12 \not\in T$.
}
By definition of subset, $S \not\subseteq T$ since there exists an element in $S$ that is not in $T$.
\\ \\
\xconclusion{$S \not\subseteq T$}
\\ \\
\end{document}