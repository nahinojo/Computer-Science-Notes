\documentclass[12pt]{article}
\usepackage{amsmath,amsfonts,amssymb,amsthm} % Math packages
\usepackage[mathscr]{euscript}
\usepackage[utf8]{inputenc}
\usepackage[T1]{fontenc}
\usepackage{titling}
\usepackage[mathscr]{euscript}
\let\euscr\mathscr \let\mathscr\relax% just so we can load this and rsfs
\usepackage[scr]{rsfso}
\usepackage{pifont}
\usepackage{geometry} % Adjust page margins
\usepackage{titlesec} % Adjust section and subsection formatting
\geometry{a4paper, left=1in, right=1in, top=1in, bottom=1in}

\newcommand{\xlist}[1]{
    \begin{itemize}
        \renewcommand{\labelitemi}{$\centerdot$}
        #1
    \end{itemize}
    \newblock
    \\ \\
}

\newcommand{\xsupposition}[1]{
    \underline{Suppositions}:
    \\ \\
    #1
    \\ \\
}

\newcommand{\xgoal}[1]{
    \underline{Goal}:
    \\ \\
    #1
    \\ \\
}

\newcommand{\xdeduction}{
    \underline{Deductions}:
    \\ \\
}

\newcommand{\xconclusion}[1]{
    \underline{Conclusion}:
    \\ \\
    #1
    \\ \\
}

\title{
  \textbf{CS-225: Discrete Structures in CS} \\
  Homework 4, Part 2
  }
\author{Noah Hinojos}
\date{\today}

\titleformat*{\subsection}{\normalsize\bfseries}

\begin{document}
\maketitle

\section*{Exercise Set 6.2}
\subsection*{13}
\xsupposition{Suppose sets $A$, $B$, and $C$ are arbitrarily chosen sets.}
\xgoal{Prove $(A-B)\cap(C-B) = (A \cap C)-B$.}
\xdeduction
By set equality, $(A-B)\cap(C-B) = (A \cap C)-B$ is true if and only if each side of the equation is a subset of the other. Hence, the following set relations must be proved:
\begin{align*}
  (A-B)\cap(C-B) \subseteq (A \cap C)-B &&\text{(i)}\\
  \vspace{1mm} \\
  \text{and \hspace*{2.5cm}} \\
  \vspace{1mm} \\
  (A \cap C)-B \subseteq (A-B)\cap(C-B) &&\text{(ii)}
\end{align*}
Proving (i) is to prove that $\forall x,$ if $x \in (A-B)\cap(C-B)$ then $x \in (A \cap C)-B$. \\
Proving (ii) is to prove that $\forall x,$ if $x \in (A \cap C)-B$ then $x \in (A-B)\cap(C-B)$. \\
\\ \\
(i) $\forall x,$ if $x \in (A-B)\cap(C-B)$ then $x \in (A \cap C)-B$.
\xlist{
  \item Suppose $x \in (A-B)\cap(C-B)$.
  \item By definition of intersection, $x \in A-B$ and $x \in C-B$.
  \item By definition of difference; $x \in A$, $x \in C$, and $x \not\in B$.
  \item By definition of intersection, $x \in A \cap C$.
  \item By definition of difference, $x \in (A \cap C) - B$.
  \item Hence, $x \in (A \cap C)-B$.
}
(ii) $\forall x,$ if $x \in (A \cap C)-B$ then $x \in (A-B)\cap(C-B)$.
\xlist{
  \item Suppose $x \in (A \cap C)-B$.
  \item By definition of difference, $x \in A \cap C$ and $x \not\in B$.
  \item By definition of intersection, $x \in A$ and $x \in C$.
  \item By definition of difference, $x \in (A-B)$ and $x \in (C-B)$.
  \item By definition of intersection, $x \in (A-B)\cap(C-B)$.
  \item Hence, $x \in (A-B)\cap(C-B)$.
}
Since (i) and (ii) are true, then by set equality the original equation $(A-B)\cap(C-B) = (A \cap C)-B$ is also true.
\\ \\
\xconclusion{Therefore, $(A-B)\cap(C-B) = (A \cap C)-B$.}

\subsection*{17}

\xsupposition{Suppose sets $A$, $B$, and $C$ are arbitrarily chosen sets. Also presume $A \subseteq B$.}
\xgoal{Prove $A \cup C \subseteq B \cup C$.}
\xdeduction
The supposition $A \subseteq B$ is equivalent to the statement that $\forall x$, if $x \in A$ then $x \in B$.
\\ \\
Proving $A \cup C \subseteq B \cup C$ is to prove that $\forall x$, if $x \in A \cup C$ then $x \in B \cup C$. 
\\ \\
$\forall x$, if $x \in A \cup C$ then $x \in B \cup C$.
\xlist{
  \item Let's take looking at the following cases.
  \item \underline {Case 1}: $x \in C$.
  \item By definition of union, $x \in B \cup C$ because $x \in C$.
  \item \underline{Case 2}: $x \in A$.
  \item By our original supposition, $x \in B$ because $x \in A$.
  \item Looking at these cases, if $x \in A$ or $x \in C$ then $x \in B$ or $x \in B \cup C$ respectively.
  \item Through definition of union, this can be rewritten as if $x \in A \cup C$ then $x \in B \cup C$.
  \item Hence, $\forall x$, if $x \in A \cup C$ then $x \in B \cup C$.
}
Since it is true that  $\forall x$, if $x \in A \cup C$ then $x \in B \cup C$; it must also be true that $A \cup C \subseteq B \cup C$. 
\\ \\
\xconclusion{Therefore, $A \cup C \subseteq B \cup C$.}

\subsection*{20}
\xsupposition{Suppose $A$, $B$, and $C$ are arbitrarily chosen sets. Also presume $A \subseteq C$ and $B \subseteq C$.}
\xgoal{Prove $A \cup B \subseteq C$.}
\xdeduction
The supposition $A \subseteq C$ is equivalent to the statement that $\forall x$, if $x \in A$ then $x \in C$. \\
The supposition $B \subseteq C$ is equivalent to the statement that $\forall x$, if $x \in B$ then $x \in C$.
\\ \\
Proving $A \cup B \subseteq C$ is to prove that $\forall x$, if $x \in A \cup B$ then $x \in C$. 
\\ \\
$\forall x$, if $x \in A \cup B$ then $x \in C$.
\xlist{
  \item Let's take looking at the following cases.
  \item \underline {Case 1}: $x \in A$.
  \item By definition of union, $x \in C$ because $x \in A$.
  \item \underline{Case 2}: $x \in B$.
  \item By our original supposition, $x \in C$ because $x \in B$.
  \item Looking at these cases, if $x \in A$ or $x \in B$ then $x \in C$.
  \item Through definition of union, this can be rewritten as if $x \in A \cup B$ then $x \in C$.
}
Since it is true that  $\forall x$, if $x \in A \cup B$ then $x \in C$; it must also be true that $A \cup B \subseteq C$.
\\ \\
\xconclusion{Therefore, $A \cup B \subseteq C$.}

\section*{Exercise Set 6.3}

\subsection*{37}
\begin{align*}
  (B^c \cup (B^c - A))^c &= B \cap (B^c - A)^c && \text{De Morgan's Law}\\
  &= B \cap (B^c \cap A)^c && \text{Set Difference Law}\\
  &= B \cap (B \cup A^c) && \text{De Morgan's Law}\\
  &= B && \text{Absorption Law}\\
\end{align*}

\subsection*{38}
\begin{align*}
  (A \cap B)^c \cap A &= (A^c \cup B^c) \cap A && \text{De Morgan's Law} \\
  &= (A^c \cap A) \cup (B^c \cap A) && \text{Distributive Law} \\
  &= \varnothing \cup (B^c \cap A) && \text{Complement Law} \\
  &= B^c \cap A && \text{Identity Law} \\
  &= A \cap B^c && \text{Communative Law} \\
  &= A - B && \text{Set Difference Law} \\
\end{align*}

\end{document}