\documentclass[12pt]{article}
\usepackage{amsmath,amsfonts,amssymb,amsthm} % Math packages
\usepackage[mathscr]{euscript}
\usepackage[utf8]{inputenc}
\usepackage[T1]{fontenc}
\usepackage{titling}
\usepackage[mathscr]{euscript}
\let\euscr\mathscr \let\mathscr\relax% just so we can load this and rsfs
\usepackage[scr]{rsfso}
\usepackage{pifont}
\usepackage{geometry} % Adjust page margins
\usepackage{titlesec} % Adjust section and subsection formatting
\geometry{a4paper, left=1in, right=1in, top=1in, bottom=1in}

\newcommand{\xlist}[1]{
    \begin{itemize}
        \renewcommand{\labelitemi}{$\centerdot$}
        #1
    \end{itemize}
    \newblock
    \\ \\
}

\newcommand{\xsupposition}[1]{
    \underline{Suppositions}:
    \\ \\
    #1
    \\ \\
}

\newcommand{\xgoal}[1]{
    \underline{Goal}:
    \\ \\
    #1
    \\ \\
}

\newcommand{\xdeductions}{
    \underline{Deductions}:
    \\ \\
}

\newcommand{\xconclusion}[1]{
    \underline{Conclusion}:
    \\ \\
    #1
    \\ \\
}

\title{
  \textbf{CS-225: Discrete Structures in CS} \\
  Homework 4, Part 1
  }
\author{Noah Hinojos}
\date{\today}

\titleformat*{\subsection}{\normalsize\bfseries}

\begin{document}
\pagenumbering{gobble}
\maketitle


\section*{Exercise Set 6.1}


\subsection*{7-a}
\xsupposition{(None)}
\xgoal{Prove $A \not\subseteq B$.}
\xdeductions
$4 \in A$
\xlist{
    \item Let $x \in A$.
    \item By definition of A, $x \in \mathbb{Z} | x = 6a + 4$.
    \item Then for $a = 0$, $x = 6(0) + 4 = 4$.
    \item Hence $4 \in A$. 
}
$4 \notin B$.
\xlist{
    \item By defintion of B, if $4 \in B$ then $18b -2 = 4$ for some integer $b$.
    \item By algebra, $4 = 18b - 2 \Rightarrow b = \frac{1}{3}$. 
    \item Yet $b$ cannot equal $\frac{1}{3}$ because $b$ is an integer.
    \item Therefore, $4 \notin B$.
}
Since $4 \in A$ and $4 \notin B$, $A \not\subseteq B$.
\\ \\
\xconclusion{$A \not\subseteq B$.}


\subsection*{7-b}
\xsupposition{Let $t$ be a particular but arbitrarily chosen element of $B$.}
\xgoal{Prove $B \subseteq A$.}
\xdeductions
$t \in \mathbb{Z} | t = 18b - 2$ for some integer $b$.
\xlist{
    \item By supposition, $t \in B$.
    \item Then by defintion of $B$, $t \in \mathbb{Z} | t = 18b - 2$ for some integer $b$.
}
$t \in A$.
\xlist{
    \item Let $a = 3b - 1$.
    \item By closure, $a$ is an integer because it is the product and summation of integers.
    \item By definition of $A$, $6a + 4 \in A$
    \item By algebra, $t = 6a + 4$:
    \begin{align*}
        6a + 4 &= 6(3b - 1) + 4 \\
        &= 18b - 6 + 4 \\
        &= 18b - 2 \\
        &= t
    \end{align*}
    \item By equality, $t \in A$ because $6a + 4 \in A$.
    \item Hence $t \in A$.
}
By definition of subset, $B \subseteq A$ because $t \in A \rightarrow t \in B$.
\\ \\
\xconclusion{Therefore, $B \subseteq A$.}
\subsection*{7-c}
By definition of set equality, $B = C$ if and only if $B \subseteq C$ and $C \subseteq B$.
\\ \\
\textbf{Proof}: $B \subseteq C$
\\ \\
\xsupposition {Let $t$ be a particular but arbitrarily chosen element of $B$.}
\xgoal{Prove $B \subseteq C$.}
\xdeductions
$t \in \mathbb{Z} | t = 18b - 2$ for some integer $b$ (See problem 7-b for proof).
\\ \\
$t \in C$.
\xlist{
    \item Let $c = b-1$.
    \item By closure, $c$ is an integer because it is the product and summation of integers.
    \item By definition of $C$, $18c + 16 \in C$
    \item By algebra, $t = 18c + 16$:
    \begin{align*}
        18c + 16 &= 18(b-1) + 16) \\
        &= 18b - 18 + 16 \\
        &= 18b - 2 \\
        &= t
    \end{align*}
    \item By equality, $t \in C$ because $18c + 16 \in C$.
    \item Hence $t \in C$.
}
By definition of subset, $B \subseteq C$ because $t \in B \rightarrow t \in C$.
\\ \\
\xconclusion{Therefore, $B \subseteq C$.}
\\ \\
Seeing that this first proof is complete, do not carry any established variables into the next proof. 
Looking ahead, the only known variable definitions should be that of $B$ and $C$, and their respective formulas, from the initial prompt.
\\ \\
\textbf{Proof}: $C \subseteq B$
\\ \\
\xsupposition {Let $t$ be a particular but arbitrarily chosen element of $C$.}
\xgoal{Prove $t \in B$.}
\xdeductions
$t \in \mathbb{Z} | t = 18c + 16$ for some integer $c$.
\xlist{
    \item By supposition, $t \in C$.
    \item Then by defintion of $C$, $t \in \mathbb{Z} | t = 18c + 16$ for some integer $c$.
}
\\ \\
$t \in B$.
\xlist{
    \item Let $b = c+1$.
    \item By closure, $b$ is an integer because it is the product and summation of integers.
    \item By definition of $B$, $18b - 2 \in B$ .
    \item By algebra, $t = 18b - 2$ :
    \begin{align*}
        18b - 2 &= 18(c+1) - 2 \\
        &= 18c + 18 - 2 \\
        &= 18c + 16 \\
        &= t
    \end{align*}
    \item By equality, $t \in B$ because $18b - 2 \in B$.
    \item Hence $t \in B$.
}
By definition of subset, $C \subseteq B$ because $t \in C \rightarrow t \in B$.
\\ \\
\xconclusion{Therefore, $C \subseteq B$.}
\\ \\
By definition of set equality, $B = C$ because $B \subseteq C$ and $C \subseteq B$.
\subsection*{26 - a}
\begin{align*}
    \bigcup_{i=1}^{4} R_i &= \bigcup_{i=1}^{4} \left[1, 1 + \frac{1}{i}\right] \\
    &= \left[1, 1 + \frac{1}{1}\right]\cup \left[1, 1 + \frac{1}{2}\right]\cup \left[1, 1 + \frac{1}{3}\right]\cup \left[1, 1 + \frac{1}{4}\right] \\
    &= \left[1, 2\right] \cup \left[1, \frac{3}{2}\right] \cup \left[1, \frac{4}{3}\right] \cup \left[1, \frac{5}{4}\right] \\
    &= \left[1, 2\right]
\end{align*}
\subsection*{26 - b}
\begin{align*}
    \bigcap_{i=1}^{4} R_i &= \bigcap_{i=1}^{4} \left[1, 1 + \frac{1}{i}\right] \\
    &= \left[1, 1 + \frac{1}{1}\right]\cap \left[1, 1 + \frac{1}{2}\right]\cap \left[1, 1 + \frac{1}{3}\right]\cap \left[1, 1 + \frac{1}{4}\right] \\
    &= \left[1, 2\right] \cap \left[1, \frac{3}{2}\right] \cap \left[1, \frac{4}{3}\right] \cap \left[1, \frac{5}{4}\right] \\
    &= \left[1, \frac{5}{4}\right]
\end{align*}
\subsection*{26 - c}
No, the sets $R_1, R_2, R_3,$... are not mutually disjoint. 
This is because all sets share the element, 1. In fact, for $i \neq j$, no set $R_i$ is mutually disjoint from $R_j$. 

\subsection*{29}
Yes. This is because all real numbers are either positive, negative, or zero. Also, there are no real numbers that are both positive and negative, and zero is neither of the two. 

\subsection*{30}
Yes. All integers can be represented by the expressions $4k$, $4k+1$,$4k+2$, and $4k+3$ where $k$ is an integer, and this through to the quotient-remainder theorem. Here, the range of the 'remainder' is $[0,3]$, i.e., all possible remainder values when dividing by 4. Since any value can be respresented by 4 times any permissible remainder value of 4, then all integers can be derived. The sets are also mutually disjoint through similar reasoning, none of the values of any expression could possibly overlap with values from any other expression.

\subsection*{33 - b}
\begin{align*}
    \mathscr{P}(\mathscr{P}(\varnothing)) &= \mathscr{P}(\{\varnothing\}) \\
    &= \{\varnothing,\{\varnothing\}\}
\end{align*}

\subsection*{33 - c}
\begin{align*}
    \mathscr{P}(\mathscr{P}(\mathscr{P}(\varnothing))) &= \mathscr{P}(\mathscr{P}(\{\varnothing\})) \\
    &= \mathscr{P}(\{\varnothing,\{\varnothing\}\}) \\
    &= \{ \varnothing, \{\{\varnothing\}\}, \{ \varnothing, \{\varnothing\}\}\}
\end{align*}

\subsection*{34 - b}
\begin{align*}
    (A_1 \cup A_2) \times A_3 &= (\{1\} \cup \{u,v\})\times \{m,n\} \\
    &= \{1, u, v\} \times \{m,n\} \\
    &= \{(1, m), (1, n), (u, m), (u, n), (v, m), (v, n)\}
\end{align*}

\subsection*{35 - d}
\begin{align*}
    (A \times B)\cap(A \times C) &= (\{ a, b \} \times \{1,2\}) \cap (\{ a, b \} \times \{2,3\}) \\
    &= \{ \{a, 1\}, \{a, 2\}, \{b, 1\}, \{b, 2\} \} \cap \{ \{a, 2\}, \{a, 3\}, \{b, 2\}, \{b, 3\} \} \\
    &= \{ \{a, 2\}, \{b, 2\} \}
\end{align*}


\end{document}
