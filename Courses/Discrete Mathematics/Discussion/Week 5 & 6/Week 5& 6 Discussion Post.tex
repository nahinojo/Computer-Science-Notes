\documentclass[12pt]{article}
\usepackage{amsmath,amsfonts,amssymb,amsthm} % Math packages
\usepackage[mathscr]{euscript}
\usepackage[utf8]{inputenc}
\usepackage[T1]{fontenc}
\usepackage{titling}
\usepackage[mathscr]{euscript}
\let\euscr\mathscr \let\mathscr\relax% just so we can load this and rsfs
\usepackage[scr]{rsfso}
\usepackage{pifont}
\usepackage{geometry} % Adjust page margins
\usepackage{titlesec} % Adjust section and subsection formatting
\geometry{a4paper, left=1in, right=1in, top=1in, bottom=1in}

\newcommand{\xlist}[1]{
    \begin{itemize}
        \renewcommand{\labelitemi}{$\centerdot$}
        #1
    \end{itemize}
    \newblock
    \\ \\
}

\newcommand{\xsupposition}{
    \underline{Suppositions}:
    \\ \\
}

\newcommand{\xgoal}{
    \underline{Goal}:
    \\ \\
}

\newcommand{\xdeduction}{
    \underline{Deductions}:
    \\ \\
}

\newcommand{\xconclusion}{
    \underline{Conclusion}:
    \\ \\
}

\newcommand{\xproof}{
    \underline{Proof}:
    \\ \\
}

\newcommand{\xbasistep}{
    \underline{Basis Step}:
    \\ \\
}

\newcommand{\xinductivehypothesis}{
    \underline{Inductive Hypothesis}:
    \\ \\
}

\newcommand{\xinductivesteps}{
    \underline{Inductive Steps}:
    \\ \\
}

\title{
  \textbf{CS-225: Discrete Structures in CS} \\
  Initial- Post Wk 5 \& 6 Problem Sets \#6 and \#15
  }
\author{Noah Hinojos}
\date{\today}

\titleformat*{\subsection}{\normalsize\bfseries}

\begin{document}
\maketitle
\section*{6}
\subsection*{Question}
Prove the following statement by mathematical induction:
$$ \forall n \in \mathbb{N}\text{,\hspace{0.5cm}}\sum_{i=1}^{n}i!(i^2 + 1) = (n+1)!n$$
\subsection*{Answer}
\xproof
$P(n) \equiv \sum_{i=1}^{n}i!(i^2 + 1) = (n+1)!n$ for every integer $n \geq 1$
\\ \\
\xbasistep
For the case $P(1)$:
\begin{align*}
    \sum_{i=1}^{1}i!(i^2 + 1) &= (1+1)1! \\
    1!(1^2 + 1) &= (2)(1) \\
    (1)(2) &= 2 \\
    2 &= 2
\end{align*}
The base case $P(1)$ holds true.
\newpage
\newblock
\\
\xinductivehypothesis
Suppose that for an arbitrary but particular integer $k$,
\\
$$P(k) \equiv \sum_{i=1}^{k}i!(i^2 + 1) = (k+1)!k\text{ for every integer }k \geq 1$$
is true.
\\ \\
\xinductivesteps
We must show that $P(k+1)$ is true. Hence we must demonstrate,
\\
$$P(k+1) \equiv \sum_{i=1}^{k+1}i!(i^2 + 1) = (k+2)!(k+1)\text{ for every integer }k \geq 1$$
\xlist{
    \item Let the expression $\sum_{i=1}^{k+1}i!(i^2 + 1)$ be referred to the left-hand side (LHS) of the equation.
    \item Let the expression $(k+2)!(k+1)$ be referred to the right-hand side (RHS) of the equation.
    \item To show that $P(k+1)$ is true, we must show that the LHS is equal to the RHS.
}
The LHS simplifies to the RHS
\xlist{
    \item Recall the LHS is $\sum_{i=1}^{k+1}i!(i^2 + 1)$
    \item By algebra and supposition of $P(k)$, the LHS can be simplified:
    \begin{align*}
        \sum_{i=1}^{k+1}i!(i^2 + 1) &= \sum_{i=1}^{k}i!(i^2 + 1) + (k+1)!((k+1)^2 + 1) \\
        &= (k+1)!k + (k+1)!(k^2 +2k + 1 + 1) \\ 
        &= (k+1)!(k^2 +3k + 2) \\ 
        &= (k+1)!(k+1)(k+2) \\ 
        &= (k+2)!(k+1) \\ 
    \end{align*}
    \item Recall the RHS is $(k+2)!(k+1)$
    \item Thus, the LHS is equal to the RHS.
}
Thus, $P(k+1) \equiv \sum_{i=1}^{k+1}i!(i^2 + 1) = (k+2)!(k+1)\text{ for every integer }k \geq 1$ was to be shown.
\\ \\
\xconclusion
Since both the basis step and the inductive step have been proved, the original expression,
$$P(n) \equiv \sum_{i=1}^{n}i!(i^2 + 1) = (n+1)!n\text{ for every integer }n \geq 1$$
is true.
\section*{15}
\subsection*{Question}
\begin{itemize}
    \item [i.] Give a recursive definition for the set of all strings of 0’s and 1’s that have more 0’s than 1’s.
    \item [ii.] Give a recursive definition for the set of all strings of $a$’s and $b$’s that begins with an $a$ and ends in a $b$.
    \item [iii.] Give a recursive definition for the set of all strings of 0’s and 1’s that are odd palindromes
\end{itemize}
\subsection*{Answer}
\subsubsection*{i.}
I - Base: The string 0 is in set $S$.
\\ \\
II - Recursion: New strings are formed according to the following rules:
\begin{itemize}
    \item [a.] If $u$ is any string in $S$, then
    \begin{center}
        $0u$ and $u0$ are strings in $S$\\
    \end{center}
    where $0u$ and $u0$ are the concatenations of $u$ and 0. Each is obtained by appending 0 to the let side of $u$ and the right side of $u$ respectively.
    \item [b.] If $u$ is any string in $S$, and $u$ contains at least two more 0's than 1's, then
    \begin{center}
        $1u$ and $u1$ are strings in $S$\\
    \end{center}
    where $1u$ and $u1$ are the concatenations of $u$ and 1. Each is obtained by appending 1 to the let side of $u$ and the right side of $u$ respectively.
\end{itemize}
III - Restriction: Every string in $S$ is obtainable from the base and the recursion.
\subsection*{ii.}
I - Base: The string $ab$ is in set $S$.
\\ \\
II - Recursion: New strings are formed according to the following rules:
\begin{itemize}
    \item [a.] If $u$ is any string in $S$, then 
    \begin{center}
        $au$ and $ub$ are strings in $S$\\
    \end{center}
    where $au$ and $ub$ are the concatenations of $u$ with $a$ and $b$ respectively.
    \item [b.] If $u$ and $v$ are any strings in $S$, then
    \begin{center}
        $uv$ is a string in $S$\\
    \end{center}
    where $uv$ is the concatenation of $u$ and $v$.
\end{itemize}
III - Restriction: Every string in $S$ is obtainable from the base and the recursion.
\subsection*{iii.}
I - Base: The strings 0 and 1 are in set $S$.
\\ \\
II - Recursion: New strings are formed according to the following rules:
\begin{itemize}
    \item [a.] If $u$ is any string in $S$, then
    \begin{center}
        $0u0$ is a string in $S$\\
    \end{center}
    where $0u0$ is the concatenation of $u$ and 0. $0u0$ is obtained by appending 0 to the left side of $u$ and the right side of $u$.
    \item [b.] If $u$ is any string in $S$, then
    \begin{center}
        $1u1$ is a string in $S$\\
    \end{center}
    where $1u1$ is the concatenation of $u$ and 1. $1u1$ is obtained by appending 1 to the left side of $u$ and the right side of $u$.
\end{itemize}
III - Restriction: Every string in $S$ is obtainable from the base and the recursion.
\end{document}