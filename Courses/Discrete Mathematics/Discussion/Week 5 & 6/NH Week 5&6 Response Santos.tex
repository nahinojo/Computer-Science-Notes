\documentclass[12pt]{article}
\usepackage{amsmath,amsfonts,amssymb,amsthm} % Math packages
\usepackage[mathscr]{euscript}
\usepackage[utf8]{inputenc}
\usepackage[T1]{fontenc}
\usepackage{titling}
\usepackage[mathscr]{euscript}
\let\euscr\mathscr \let\mathscr\relax% just so we can load this and rsfs
\usepackage[scr]{rsfso}
\usepackage{pifont}
\usepackage{geometry} % Adjust page margins
\usepackage{titlesec} % Adjust section and subsection formatting
\geometry{a4paper, left=1in, right=1in, top=1in, bottom=1in}

\newcommand{\xlist}[1]{
    \begin{itemize}
        \renewcommand{\labelitemi}{$\centerdot$}
        #1
    \end{itemize}
    \newblock
    \\ \\
}

\newcommand{\xsupposition}{
    \underline{Suppositions}:
    \\ \\
}

\newcommand{\xgoal}{
    \underline{Goal}:
    \\ \\
}

\newcommand{\xdeduction}{
    \underline{Deductions}:
    \\ \\
}

\newcommand{\xconclusion}{
    \underline{Conclusion}:
    \\ \\
}

\newcommand{\xproof}{
    \underline{Proof}:
    \\ \\
}

\newcommand{\xbasistep}{
    \underline{Basis Step}:
    \\ \\
}

\newcommand{\xinductivehypothesis}{
    \underline{Inductive Hypothesis}:
    \\ \\
}

\newcommand{\xinductivesteps}{
    \underline{Inductive Steps}:
    \\ \\
}

\title{
  \textbf{CS-225: Discrete Structures in CS} \\
  Week 5-6, Response to Praj - Problem 16 
  }
\author{Noah Hinojos}
\date{\today}

\titleformat*{\subsection}{\normalsize\bfseries}

\begin{document}
\maketitle
\section*{Response}
\subsection*{Proof Variable Mismatch}
Your proof states for $P(k)$, but the equivalent statement uses the variable $n$. 
And, you constrict the domain using $k$ as well by stating $k \geq 64$.
This is a clear mismatch between $n$ and $k$. The correct format should use $n$ in all circumstances, such as:
\\ \\
$ P(n) \equiv $ a postage of $n$ cents can be formed using just 5-cent and 17-cent stamps, where $n$ is an integer and $n\geq 64$
\\ \\
Other than that, great work! I'd like to point out a similar problem in the textbook, Proposition 5.3.1. They prove $P(k+1)$ in the same style of a postage problem. However, they use weak induction instead of strong induction. May be worth checking out.
\end{document}
