\documentclass[12pt]{article}
\usepackage{amsmath,amsfonts,amssymb,amsthm} % Math packages
\usepackage[mathscr]{euscript}
\usepackage[utf8]{inputenc}
\usepackage[T1]{fontenc}
\usepackage{titling}
\usepackage[mathscr]{euscript}
\let\euscr\mathscr \let\mathscr\relax% just so we can load this and rsfs
\usepackage[scr]{rsfso}
\usepackage{pifont}
\usepackage{geometry} % Adjust page margins
\usepackage{titlesec} % Adjust section and subsection formatting
\geometry{a4paper, left=1in, right=1in, top=1in, bottom=1in}

\newcommand{\xlist}[1]{
    \begin{itemize}
        \renewcommand{\labelitemi}{$\centerdot$}
        #1
    \end{itemize}
    \newblock
    \\ \\
}

\newcommand{\xsupposition}{
    \underline{Suppositions}:
    \\ \\
}

\newcommand{\xgoal}{
    \underline{Goal}:
    \\ \\
}

\newcommand{\xdeduction}{
    \underline{Deductions}:
    \\ \\
}

\newcommand{\xconclusion}{
    \underline{Conclusion}:
    \\ \\
}

\newcommand{\xproof}{
    \underline{Proof}:
    \\ \\
}

\newcommand{\xbasistep}{
    \underline{Basis Step}:
    \\ \\
}

\newcommand{\xinductivehypothesis}{
    \underline{Inductive Hypothesis}:
    \\ \\
}

\newcommand{\xinductivesteps}{
    \underline{Inductive Steps}:
    \\ \\
}

\title{
  \textbf{CS-225: Discrete Structures in CS} \\
  Week 5-6, Response to Praj - Problem 16 
  }
\author{Noah Hinojos}
\date{\today}

\titleformat*{\subsection}{\normalsize\bfseries}

\begin{document}
\maketitle
\section*{Response}
\subsection*{Syntactical Issue}
You define your inductive hypothesis as follows:
$$\forall(k \geq 7)\in \mathbb{Z}, k! \geq 3^k + 2^k$$
There's nothing incorrect here in terms of the problem. 
However, I'm quite certain this is syntactically invalid, specifically this expression:
$$\forall(k \geq 7) \in \mathbb{Z}$$
The universal quantifier $\forall$ must be succeeded by the variable itself, as in $\forall x$.
The expression $k \geq 7$ is technically a predicate and therefore must be distinct from the quantifier.
\\ \\
However, textbook doesn't provide a set of air-tight syntactical rules for how to format these expressions. 
Section 3.2 does have plentiful examples that agree with my criticism, yet, these examples are still arguably informal as they often uses english words to connect expressions. 
I'm therefore hesitant to state your syntax is truly \textit{wrong} (yet that's not to say it can be made more right). 
Anyways, here's my proposed solution based on the textbook examples:
$$\forall k \in \mathbb{Z},k! \geq 3^k + 2^k \text{ where } k \geq 7$$
Or better yet, applying our knowledge of propisitional logic:
$$\forall k \in \mathbb{Z},(k! \geq 3^k + 2^k) \wedge (k \geq 7)$$
Also, I'd like to add that if you feel like I'm being pedantic that's totally understandable (I kinda am). Just note I'm really just trying to find ways for you to to improve.
\subsection*{Readability}
There are a few ways you can make your work more readable. Sure it's not part of the grade but your work is a bit of a challenge to follow in its current state. I know I know, another pedantic criticism, but I do have a couple ways you can improve:
\xlist{
  \item You state "Starting Point" at the header of the problem, yet, you bullet point everything else. Are all these points within Starting Point? Consider placing the other headers (Inductive Hypothesis/Step, Work, Conclusion) on the far right like you did with Starting Point.
  \item When you write mathematical expressions, the bullet points make a it a bit challenging to follow especially when there's multiple points of work at differing indentation levels. Consider not having any bullet points for math expression and just center it in the page. I did this in the criticism above and would like to think it looks much nicer. 
}
If you can't appreciate the niceness of a well-formatted solution for its own sake, then please consider it for the sake of grading. Difficult to read work can be more difficult to grade accurately.
\\ \\
Anyways, great job on the problem itself! No issues found, just pedantic syntactical improvements :)
 

\end{document}

