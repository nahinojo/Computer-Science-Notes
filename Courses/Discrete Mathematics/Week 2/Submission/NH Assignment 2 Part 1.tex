\documentclass[12pt]{article}

\usepackage{amsmath,amsfonts,amssymb,amsthm} % Math packages
\usepackage[utf8]{inputenc}
\usepackage[T1]{fontenc}
\usepackage{titling}
\usepackage{geometry} % Adjust page margins
\usepackage{titlesec} % Adjust section and subsection formatting
\geometry{a4paper, left=1in, right=1in, top=1in, bottom=1in}

\title{
  \textbf{CS-225: Discrete Structures in CS} \\
  Homework 2, Part 1
  }
\author{Noah Hinojos}
\date{\today}

\titleformat*{\subsection}{\normalsize\bfseries}

\begin{document}

\maketitle

\section*{Exercise Set 3.1}

\subsection*{29 - b}
There exists a geometric figure that is a rectangle and is a square. 
\textbf{True}, all squares are rectangles.

\subsection*{29 - c}
There exists a geometric figure that is a rectangle and is a not square.
\textbf{True}, a \textit{long} rectangle is not a square since not all side lengths are equal.

\subsection*{30 - a}
There exists an integer such that it is prime and it is not odd.
\textbf{True}, the number 2 is prime and not odd.

\subsection*{30 - c}
There exists an integer such that it is odd and is is a perfect square.
\textbf{True}, the number 25 is odd and is a perfect square.

\section*{Exercise Set 3.2}

\subsection*{8***}
\underline{Informal Negation}: There exists a simple solution to life's problems.

\subsection*{14}
\underline{Informal Negation}: For all real numbers $x_1$ and $x_2$, if $x_1=x_2$ then $x_1^2 = x_2^2$.

\subsection*{31}
\underline{Original Statement}: $\forall$ integer $n$, if $n$ is divisible by 6, then $n$ is divisible by 2 and $n$ is divisible by $3$.
\\ \\
Let $n$ be a variable and $\mathbb{Z}$ be the set of all integers. \\
Let $P(n)$ be the statement "if $n$ is divisible by 6." \\
Let $Q(n)$ be the statement "if $n$ is divisible by 2." \\
Let $R(n)$ be the statement "if and $n$ is divisible by 3."
\\ \\
The Original Statement can then be formulated as $\forall n \in \mathbb{Z}, P(n) \rightarrow (Q(n) \wedge R(n))$
\\ \\
Converse: $\forall n \in \mathbb{Z}, (Q(n) \wedge R(n)) \rightarrow P(n)$ \\
Inverse: $\forall n \in \mathbb{Z}, \sim P(n) \rightarrow \ (\sim Q(n) \ \vee \sim R(n))$ \\
Contrapositive: $\forall n \in \mathbb{Z}, (\sim Q(n) \ \vee \sim R(n)) \rightarrow \sim P(n)$
\\ \\
Rewriting these verbally, 
\\ \\
\underline{Converse}: $\forall$ integer $n$, if $n$ is divisible by 2 and  $n$ is divisible by 3, then $n$ is divisible by 6. This statement is \textbf{true}.
\\ \\
\underline{Inverse}: $\forall$ integer $n$, if $n$ is not divisible by 6, then neither $n$ is divisible by 2 nor $n$ is divisible by 3. This statement is \textbf{false} for the counterexample $n = 2$.
\\ \\
\underline{Contrapositive}: $\forall$ integer $n$, if neither $n$ is divisible by 2 nor $n$ is divisible by 3, then $n$ is not divisible by 6. This statement is \textbf{true}.

\subsection*{44}
If a polygon is a square, then it has four sides.

\subsection*{48}
\underline{Original Statement}: Being a polynomial is not a sufficient condition for a function to have a real root.
\\ \\
Let $f$ be a variable and $F$ be the set of all functions. \\
Let $P(f)$ be the statement "the function is a polynomial." \\
Let $Q(f)$ be the statement "the function has a real root."
\\ \\
The Original Statement can then be formulated as $\sim(\forall f \in F,P(f) \rightarrow Q(f))$. \\
\\ \\
Reformulating to not use the confitional symbology,
\\ \\
\begin{align*}
  \sim(\forall f \in F,P(f) \rightarrow Q(f)) &\equiv \exists f \in F, P(f) \ \wedge \sim Q(f)
\end{align*}
Rewriting this verbally,
\\ \\
\underline{Answer}: There exists a function such that the function is a polynomial and the function does not have a real root. 

\subsection*{49}
\underline{Original Statement}: The absence of error messages during translation of a computer program is only a necessary and not a sufficient condition for reasonable [program] correctness.
\\ \\
Let $x$ be a variable and $C$ be the set of all computer programs. \\
Let $P(x)$ be the statement "there is an absence of error messages during translation" \\
The $Q(x)$ be the statement "there is reasonable correctness." 
\\ \\
The Original Statement is the conjunction of the both following, \\
Necessary: $\forall x \in C,\sim P(x) \rightarrow \ \sim Q(x) \equiv \forall x \in C,P(x) \ \vee \sim Q(x)$ \\
Not Sufficient: $\sim(\forall x \in C,P(x) \rightarrow Q(x)) \equiv \exists x \in C,P(x) \ \wedge \sim Q(x)$ \\
The Original Statement can therefore be formulated as, 
$$(\forall x \in C,P(x) \ \vee \sim Q(x)) \wedge (\exists x \in C,P(x) \ \wedge \sim Q(x))$$
\\ \\
Rewriting this verbally,
\\ \\
\underline{Answer}: For all computer programs, there is an absence of error messages during translation or there is no reasonable correctness.
And, there exists a computer program such that there is an absence of error messages during translation and there is no reasonable correctness.

\section*{Canvas Problem}
\subsection*{a}
$\forall x \in D, C(x) \rightarrow A(x)$

\subsection*{b}
$\forall x \in D, C(x) \rightarrow (F(x) \wedge H(x))$

\subsection*{c}
$\forall x \in D,\sim(A(x) \rightarrow L(x))$

\subsection*{d}
$\exists x \in D, C(x) \rightarrow (\sim H(x) \ \vee \sim F(x))$

\subsection*{e}
$\forall x \in D, C(x) \rightarrow A(x)$

\end{document}