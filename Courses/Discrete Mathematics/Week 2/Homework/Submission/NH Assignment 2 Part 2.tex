\documentclass[12pt]{article}

\usepackage{amsmath,amsfonts,amssymb,amsthm} % Math packages
\usepackage[utf8]{inputenc}
\usepackage[T1]{fontenc}
\usepackage{titling}
\usepackage{geometry} % Adjust page margins
\usepackage{titlesec} % Adjust section and subsection formatting
\geometry{a4paper, left=1in, right=1in, top=1in, bottom=1in}

\title{
  \textbf{CS-225: Discrete Structures in CS} \\
  Homework 2, Part 2
  }
\author{Noah Hinojos}
\date{\today}

\titleformat*{\subsection}{\normalsize\bfseries}

\begin{document}
\maketitle

\section*{Canvas Problems}

\subsection*{1}

For the sequence, \\
$$\frac{1}{4}, \frac{3}{12}, \frac{5}{36}, \frac{7}{108}, \frac{9}{324},...$$
\\
The corresponding sequence formula is, \\
$$a_n = \frac{2n+1}{4\cdot3^n} \ \ \ \ \text{for every integer } n \geq 0$$
\\ \\ \\
For the sequence,
$$0, -\frac{1}{3}, \frac{2}{4}, -\frac{3}{5}, \frac{4}{6}, -\frac{5}{7},...$$
\\
The corresponding sequence formula is, \\
$$a_n = \frac{(-1)^n \cdot n}{n+2} \ \ \ \ \text{ for every integer } n \geq 0$$

\subsection*{2}
Given,
$$\sum_{k=1}^{n} \frac{2k+1}{k^2(k+1)^2}$$
\\
We can use partial fractions to simplify the inside expression. Lets start by assuming,
$$\frac{2k+1}{k^2(k+1)^2} = \frac{A}{k^2} + \frac{B}{(k+1)^2}$$
\\
Now solving for $A$ and $B$,
\begin{align*}
  \frac{2k+1}{k^2(k+1)^2} &= \frac{A}{k^2} + \frac{B}{(k+1)^2}\\ \\
  \frac{2k+1}{k^2(k+1)^2} &= \frac{A(k+1)^2}{k^2(k+1)^2} + \frac{Bk^2}{k^2(k+1)^2}\\ \\
  2k+1 &= A(k+1)^2 + Bk^2\\
  2k+1 &= A(k^2+2k+1) + Bk^2\\
  2k+1 &= Ak^2+2Ak+A + Bk^2\\
  2k+1 &= (A+B)k^2+A(2k+1)\\
\end{align*}
\\
Solving for $A$,
\begin{align*}
  2k+1 &= A(2k+1) \\
  A &= 1
\end{align*}
\\
Solving for $B$ when $A=1$,
\begin{align*}
  0 &= (A+B)k^2 \\
  0 &= (1+B)k^2 \\
  0 &= 1+B \\
  B &= -1
\end{align*}
\\
Substituting $A=1$ and $B=-1$ into the original expression,
\begin{align*}
  \frac{2k+1}{k^2(k+1)^2} &= \frac{A}{k^2} + \frac{B}{(k+1)^2} \\ \\
  \frac{2k+1}{k^2(k+1)^2} &= \frac{1}{k^2} - \frac{1}{(k+1)^2} \\ \\
\end{align*}
Substituting this new expression into the original summation formula,
\begin{align*}
  \sum_{k=1}^{n} \frac{2k+1}{k^2(k+1)^2} &= \sum_{k=1}^{n} \left( \frac{1}{k^2} - \frac{1}{(k+1)^2} \right) \\ \\
  &= \left( \frac{1}{1} - \frac{1}{4} \right) + \left( \frac{1}{4} - \frac{1}{9} \right) 
  + \left( \frac{1}{9} - \frac{1}{16} \right) +...+ \left( \frac{1}{n^2} - \frac{1}{(n+1)^2} \right) \\ \\
  &=  \frac{1}{1} + \left( \frac{1}{4} - \frac{1}{4} \right) + \left( \frac{1}{9} - \frac{1}{9} \right) +...
  + \left( \frac{1}{n^2} - \frac{1}{n^2} \right) - \frac{1}{(n+1)^2}\\ \\
  &= 1 - \frac{1}{(n+1)^2}
\end{align*}

\subsection*{3}
\begin{align*}
  \sum_{i=30}^{500}\left(10i-\frac{5}{2}\right) &= \sum_{i=30}^{500}10i - \sum_{i=30}^{500}\frac{5}{2}\\ \\
  &= 10\sum_{i=30}^{500}i - \frac{5}{2}(500-30+1)\\ \\
  &= 10\sum_{i=30}^{500}i - \frac{5}{2}(500-30+1) \\ \\
  &= 10\left( \sum_{i=1}^{500}i - \sum_{i=1}^{29}i \right) - \frac{5}{2}(500-30+1) \\ \\
  &= 10\left( \frac{500(500+1)}{2} - \frac{29(29+1)}{2} \right) - \frac{5}{2}(500-30+1) \\ \\
\end{align*}

\subsection*{4}
\begin{align*}
  \sum_{j=0}^{200}(200j^2 - (-20)^j) &= \sum_{j=0}^{200}200j^2 - \sum_{j=0}^{200}(-20)^j\\ \\
  &= 200\sum_{j=0}^{200}j^2 - \sum_{j=0}^{200}(-20)^j\\ \\
  &= 200\left(\frac{200(200+1)(2\cdot200+1)}{6}\right) - \left(\frac{(-20)^{200+1}}{-20-1} \right)  \\
\end{align*}

\subsection*{5}
\begin{align*}
  4 \sum_{k=1}^{15}(14k^2 + 9) + 6 \sum_{k=1}^{15}(15k^2-7) &= \sum_{k=1}^{15}(56k^2 - 36) + \sum_{k=1}^{15}(90k^2-42) \\ \\
  &= \sum_{k=1}^{15}(56k^2 - 36 + 90k^2-42) \\ \\
  &= \sum_{k=1}^{15}(146k^2 - 78) \\ \\
  &= \sum_{k=1}^{15}146k^2 - \sum_{k=1}^{15}78 \\ \\
  &= 146\sum_{k=1}^{15}k^2 - 78\sum_{k=1}^{15}1 \\ \\
  &= 146\left(\frac{15(15+1)(2\cdot 15+1)}{6}\right) - 78(15-1+1)\\
\end{align*}
\end{document}