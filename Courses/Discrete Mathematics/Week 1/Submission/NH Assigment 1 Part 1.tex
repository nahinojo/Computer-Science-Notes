\documentclass[12pt]{article}

\usepackage{amsmath,amsfonts,amssymb,amsthm} % Math packages
\usepackage[utf8]{inputenc}
\usepackage[T1]{fontenc}
\usepackage{geometry} % Adjust page margins
\usepackage{titlesec} % Adjust section and subsection formatting
\geometry{a4paper, left=1in, right=1in, top=1in, bottom=1in}

\title{\textbf{Assignment 1, Part 1: The Logic of Compound Statements}}
\author{Noah Hinojos}
\date{\today}

\titleformat*{\subsection}{\normalsize\bfseries}

\begin{document}

\maketitle

\section*{Exercise Set 2.1}

\subsection*{5 - b}
"She is a mathematics major." is not a proposition. The statement's truth value depends on who "she" is referring to. If the pronoun refers to math major, then it is true. Otherwise, it is false. The statement may be either true or false in this case.

\subsection*{5 - c}
$128 = 2^6$ is a proposition.

\subsection*{5 - d}
$x = 2^6$ is not a proposition. The statement's truth value depends on the value of $x$. Since $x$ may or may not be $128$, the equation may be either true or false.

\subsection*{8 - c}
$ \sim h \ \wedge \sim w \ \wedge \sim s$

\subsection*{10 - e}
$ \sim p \vee (q \wedge r)$

\subsection*{30}
The dollar is not at an all-time high or the stock is not at a record low.

\subsection*{37}
$(x < -7)\vee(x \geq 0)$

\subsection*{39}
$(num\_orders \geq 50$ or $num\_instock \leq 300)$ and 
\\ $(num\_orders < 50$ or $num\_orders \geq 75$ or $num\_instock \leq 500)$
\subsection*{42}

\begin{tabular}{c c c|c c|c|c|c}
    $p$ & $q$ & $r$ & $\sim p$ & $\sim q$ & $\sim p \wedge q$ & $q \wedge r$ & $((\sim p \wedge q)\wedge(q \wedge r))\wedge \sim q$ \\
    \hline
    T&T&T&F&F&F&T&F \\
    T&T&F&F&F&F&F&F \\
    T&F&T&F&T&F&F&F \\
    T&F&F&F&T&F&F&F \\
    F&T&T&T&F&T&T&F \\
    F&T&F&T&F&T&F&F \\
    F&F&T&T&T&F&F&F \\
    F&F&F&T&T&F&F&F \\
\end{tabular}
\\ \\ \\
The formula $((\sim p \wedge q)\wedge(q \wedge r))\wedge \sim q$ is a \textbf{contradiction} because it is always false.
\subsection*{49}
\begin{enumerate}
    \item[a.] Commutative Law
    \item[b.] Distributive Law
    \item[c.] Negation Law
    \item[d.] Identity Law
\end{enumerate}
\subsection*{54}
\begin{align*}
    p &\equiv (p \wedge (\sim(\sim p \vee q)))\vee(p \wedge q) \\
    &\equiv (p \wedge (p \ \wedge \sim q)))\vee(p \wedge q) && \text{De Morgan's Law}\\
    &\equiv ((p \wedge p) \ \wedge \sim q)\vee(p \wedge q) && \text{Associative Law}\\
    &\equiv (p \ \wedge \sim q)\vee(p \wedge q) && \text{Idempotent Law}\\
    &\equiv p \wedge (\sim q \vee q) && \text{Distributive Law}\\
    &\equiv p \wedge \textbf{t} && \text{Negation Law}\\
    &\equiv p && \text{Identity Law}
\end{align*}

\section*{Canvas Problem}
\begin{align*}
    \sim p &\equiv ((\sim p \wedge q) \vee (\sim p \ \wedge \sim q)) \vee (\sim p \wedge q) \\
    &\equiv (\sim p \ \wedge \sim q) \vee ((\sim p \wedge q) \vee (\sim p \wedge q)) && \text{Commutative Law} \\
    &\equiv ((\sim p \wedge q) \vee (\sim p \wedge q)) \vee (\sim p \ \wedge \sim q) && \text{Associative Law} \\
    &\equiv (\sim p \wedge q) \vee (\sim p \ \wedge \sim q) && \text{Idempotent Law} \\
    &\equiv \sim p \wedge (q \ \vee \sim q) && \text{Distributive Law} \\
    &\equiv \sim p \wedge \textbf{t} && \text{Negation Law} \\
    &\equiv \sim p &&\text{Identity Law} \\
\end{align*}

\end{document}
