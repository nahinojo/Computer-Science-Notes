\documentclass[12pt]{article}
\usepackage{amsmath,amsfonts,amssymb,amsthm} % Math packages
\usepackage[mathscr]{euscript}
\usepackage[utf8]{inputenc}
\usepackage[T1]{fontenc}
\usepackage{titling}
\usepackage[mathscr]{euscript}
\let\euscr\mathscr \let\mathscr\relax% just so we can load this and rsfs
\usepackage[scr]{rsfso}
\usepackage{pifont}
\usepackage{geometry} % Adjust page margins
\usepackage{titlesec} % Adjust section and subsection formatting
\geometry{a4paper, left=1in, right=1in, top=1in, bottom=1in}

\newcommand{\xlist}[1]{
    \begin{itemize}
        \renewcommand{\labelitemi}{$\centerdot$}
        #1
    \end{itemize}
    \newblock
    \\ \\
}

% Deductive Proofs %
\newcommand{\xsupposition}[1]{
    \underline{Suppositions}:
    \\ \\
    #1
    \\ \\
}
\newcommand{\xgoal}[1]{
    \underline{Goal}:
    \\ \\
    #1
    \\ \\
}
\newcommand{\xdeductions}{
    \underline{Deductions}:
    \\ \\
}
\newcommand{\xconclusion}[1]{
    \underline{Conclusion}:
    \\ \\
    #1
    \\ \\
}

% Inductive Proofs %
\newcommand{\xproof}[1]{
    \underline{Proof}:
    \\ \\
    #1
    \\ \\
}
\newcommand{\xbasisstep}{
    \underline{Basis Step}:
    \\ \\
}
\newcommand{\xinductivehypothesis}{
    \underline{Inductive Hypothesis}:
    \\ \\
}
\newcommand{\xinductivestep}{
    \underline{Inductive Steps}:
    \\ \\
}

\title{
  \textbf{CS-225: Discrete Structures in CS} \\
  Homework 5, Part 2
  }
\author{Noah Hinojos}
\date{\today}

\titleformat*{\subsection}{\normalsize\bfseries}

\begin{document}
\pagenumbering{gobble}
\maketitle


\section*{Exercise Set 5.4}
\subsection*{3}
\xproof{
    Given the sequence $c_0, c_1, c_2$... defined below:
    \begin{align*}
        c_0 &= 2, c_1 = 2, c_2 = 6, \hspace{8cm}\\
        c_k &= 3c_{k-3}\text{ for every integer } k \geq 3. \hspace{8cm}
    \end{align*}
    Let $P(n) \equiv c_n$ is even for each integer $n \geq 0$.
}
\xbasisstep
For the case $P(0)$:
\begin{align*}
    c_0 = 2 \hspace{14cm}
\end{align*}
Since $c_0 = 2$ is even, $P(0)$ holds true.
\\ \\
For the case $P(1)$:
\begin{align*}
    c_1 = 2 \hspace{14cm}
\end{align*}
Since $c_1 = 2$ is even, $P(1)$ holds true.
\\ \\
For the case $P(2)$:
\begin{align*}
    c_2 = 6 \hspace{14cm}
\end{align*}
Since $c_2 = 6$ is even, $P(2)$ holds true.
\\ \\
\xinductivehypothesis
Suppose that for an arbitrary but particular integer $k$, such that $k \geq 0$,
\begin{center}
    $P(i) \equiv c_i$ is even for each integer $k \geq i \geq 0$
\end{center}
\xinductivestep
We must show that $P(k+1)$ is true. Hence we must demonstrate,
\begin{center}
    $P(k+1) \equiv c_{k+1}$ is even for each integer $k + 1 \geq 0$
\end{center}
\newblock
\\ \\
$P(k+1)$ holds true.
\xlist{
    \item For $P(k+1)$, $c_{k+1} = 3c_{k-2}$.
    \item By inductive hypotheses, $c_{k-2}$ is even.
    \item $c_{k+1}$ must also be even because it is the product of 3 and an even number $c_{k-2}$.
    \item Since $c_{k+1}$ is even, $P(k+1)$ holds true.
    \item Hence, $P(k+1)$ holds true.
}
Thus, it was shown to be true that $P(k+1) \equiv c_{k+1}$ is even for each integer $k + 1 \geq 0$.
\\ \\
\xconclusion{
    Since both the basis and inductive step have been proved, the original expression,
    \begin{center}
        $P(n) \equiv c_n$ is even for each integer $n \geq 0$.
    \end{center}
    is true.
}


\subsection*{8 - a}
\xproof{
    Given the sequence $h_0, h_1, h_2$... defined below:
    \begin{align*}
        h_0 &= 1, h_1 = 2, h_2 = 3, \hspace{10cm} \\
        h_k &= h_{k-1}+h_{k-2}+h_{k-3}\text{ for every integer } k \geq 3.
    \end{align*}
    Let $P(n) \equiv h_n \leq 3^n $ for every integer $n \geq 0$.
}
\xbasisstep
For the case $P(0)$:
\begin{align*}
    h_0 = 1 \hspace{14cm}
\end{align*}
Since $1 \leq 3^0$, $P(0)$ holds true.
\\ \\
For the case $P(1)$:
\begin{align*}
    h_1 = 2 \hspace{14cm}
\end{align*}
Since $2 \leq 3^1$, $P(1)$ holds true.
\\ \\
For the case $P(2)$:
\begin{align*}
    h_2 = 3 \hspace{14cm}
\end{align*}
Since $3 \leq 3^2$, $P(2)$ holds true.
\\ \\
\xinductivehypothesis
Suppose that for an arbitrary but particular integer $k$, such that $k \geq 0$,
\begin{center}
    $P(i) \equiv h_i \leq 3^i $ for each integer $k \geq i \geq 0$
\end{center}
\xinductivestep
We must show that $P(k+1)$ is true. Hence we must demonstrate,
\begin{center}
    $P(k+1) \equiv h_{k+1} \leq 3^{k+1}$ for each integer $k + 1 \geq 0$
\end{center}
\newblock
\\ \\
$P(k+1)$ holds true.
\xlist{
    \item For $P(k+1)$:
    \begin{align*}
        h_{k+1} &= h_{k}+h_{k-1}+h_{k-2} \hspace{9cm} \\
        &\leq 3^{k}+ 3^{k-1} + 3^{k-2} \\
        &\leq 3^k+3^k+3^k \\
        &\leq 3^{k+1}
    \end{align*}
    \item Hence, $P(k+1)$ holds true.
}
Thus, it was shown to be true that $P(k+1) \equiv h_{k+1} \leq 3^{k+1}$ for each integer $k + 1 \geq 0$.
\\ \\
\xconclusion{
    Since both the basis and inductive step have been proved, the original expression,
    \begin{center}
        $P(n) \equiv h_n \leq 3^n $ for every integer $n \geq 0$.
    \end{center}
    is true.
}


\subsection*{9}
\xproof{
    Given the sequence $a_0, a_1, a_2$... defined below:
    \begin{align*}
        a_1 &= 1, a_2 = 3 \hspace{12cm}\\
        a_k &= a_{k-1}+a_{k-2}\text{ for every integer } k \geq 3.
    \end{align*}
    Let $P(n) \equiv a_n \leq \left(\frac{7}{4}\right)^n $ for every integer $n \geq 1$.
}
\xbasisstep
For the case $P(1)$:
\begin{align*}
    a_1 = 1 \hspace{14cm}
\end{align*}
Since $1 \leq \frac{7}{4}$, $P(1)$ holds true.
\\ \\
For the case $P(2)$:
\begin{align*}
    a_2 = 3 \hspace{14cm}
\end{align*}
Since $3 \leq \frac{49}{16}$, $P(2)$ holds true.
\\ \\
\xinductivehypothesis
Suppose that for an arbitrary but particular integer $k$, such that $k \geq 1$,
\begin{center}
    $P(i) \equiv a_i \leq \left(\frac{7}{4}\right)^i $ for each integer $k \geq i \geq 1$
\end{center}
\xinductivestep
We must show that $P(k+1)$ is true. Hence we must demonstrate,
\begin{center}
    $P(k+1) \equiv a_{k+1} \leq \left(\frac{7}{4}\right)^{k+1}$ for each integer $k + 1 \geq 1$
\end{center}
$P(k+1)$ holds true.
\xlist{
    \item For $P(k+1)$:
    \begin{align*}
        a_{k+1} &= a_k + a_{k-1} \\
        &\leq \left(\frac{7}{4}\right)^k + \left(\frac{7}{4}\right)^{k-1} \\
        &\leq \left(\frac{7}{4}\right)^k + \left(\frac{7}{4}\right)^k \\
        &\leq \left(\frac{7}{4}\right)^{k+1}\\
    \end{align*}
    \item 
}
Thus, it was shown to be true that $P(k+1) \equiv a_{k+1} \leq \left(\frac{7}{4}\right)^{k+1}$ for each integer $k + 1 \geq 1$.
\xconclusion{
    Since both the basis and inductive step have been proved, the original expression,
    \begin{center}
        $P(n) \equiv a_n \leq \left(\frac{7}{4}\right)^n $ for every integer $n \geq 1$.
    \end{center}
    is true.
}
\subsection*{Canvas Problem}
\xproof{
    Let $P(n) \equiv$ a postage of $n$ cents can be formed using just $5$-cent and $7$-cent stamps, where $n$ is any integer $\geq 24$. 
}
\xbasisstep
The case $P(24)$ where the postage is 24 cents can be formed using two $5$-cent and two $7$-cent stamps. \\ \\
\xinductivehypothesis
Suppose that for an arbitrary but particular integer $k$, such that $k \geq 24$,
\begin{center}
    A postage of $k$ cents can be formed using just $5$-cent and $7$-cent stamps.
\end{center}
\xinductivestep
We must show that $P(k+1)$ is true. Hence we must demonstrate,
\begin{center}
    A postage of $k+1$ cents can be formed using just $5$-cent and $7$-cent stamps.
\end{center}
\newblock
\\ \\
There are six total cases for $P(k)$ of how a postage can be represented:
\xlist{
    \item Case 1: The postage is formed using no $5$-cent stamps.
    \item Case 2: The postage is formed using one $5$-cent stamp.
    \item Case 3: The postage is formed using two $5$-cent stamps.
    \item Case 4: The postage is formed using three $5$-cent stamps.
    \item Case 5: The postage is formed using four $5$-cent stamps.
    \item Case 6: The postage is formed using five or more $5$-cent stamps.
}
We will exhaustively demonstrate how each of the six cases of $P(k)$ can be easily transformed for $P(k+1)$.
\\ \\
\newblock
\\ \\ 
Case 1: The postage is formed using no $5$-cent stamps.
\xlist{
    \item In Case 1 of $P(k)$, the postage is formed using no $5$-cent stamps.
    \item Hence, the postage is made up of at least four $7$-cent stamps. These four stamps summate to $28$ cents.
    \item To convert to $P(k+1)$, replace these four stamps with three $5$-cent stamps and two $7$-cent stamps. This new replacement summates to $29$ cents.  
}
\\ \\
Case 2: The postage is formed using one $5$-cent stamp.
\xlist{
    \item In Case 2 of $P(k)$, the postage is formed using one $5$-cent stamp.
    \item Hence, the postage is made up of at least three $7$-cent stamps. These four stamps summate to $26$ cents.
    \item To convert to $P(k+1)$, replace the these four with one $7$-cent stamp and four $5$-cent stamps. This new replacement summates to $27$ cents. 
}
\\ \\
Case 3: The postage is formed using two $5$-cent stamps.
\xlist{
    \item In Case 3 of $P(k)$, the postage is formed using two $5$-cent stamps.
    \item Hence, the postage is made up of at least two $7$-cent stamps. These four stamps summate to $24$ cents.
    \item To convert to $P(k+1)$, replace the these four with five $5$-cent stamps. This new replacement summates to $25$ cents.
}
\\ \\
Case 4: The postage is formed using three $5$-cent stamps.
\xlist{
    \item In Case 4 of $P(k)$, the postage is formed using three $5$-cent stamps.
    \item Hence, the postage is made up of at least two $7$-cent stamp. These four stamps summate to $29$ cents.
    \item To convert to $P(k+1)$, replace the these four with six $5$-cent stamps. This new replacement summates to $30$ cents.
}
\\ \\
Case 5: The postage is formed using four $5$-cent stamps.
\xlist{
    \item In Case 5 of $P(k)$, the postage is formed using four $5$-cent stamps.
    \item Hence, the postage is made up of at least one $7$-cent stamps. These four stamps summate to $27$ cents.
    \item To convert to $P(k+1)$, replace the these four with four $7$-cent stamps. This new replacement summates to $28$ cents. 
}
\\ \\
Case 6: The postage is formed using five or more $5$-cent stamps.
\xlist{
    \item In Case 6 of $P(k)$, the postage is formed using five or more $5$-cent stamps. Let's consider the five $5$-cent stamps, which summate to $25$ cents.
    \item To convert to $P(k+1)$, replace the these five stamps stamps with three $7$-cent stamps and one $5$-cent stamps. This new replacement summates to $26$ cents. 
}
Thus, it was shown to be true that $P(k+1) \equiv$ a postage of $n$ cents can be formed using just $5$-cent and $7$-cent stamps, where $n$ is any integer $\geq 24$. \\ \\
\xconclusion{
    Since both the basis and inductive step have been proved, the original expression, \\ \\
    $P(n) \equiv$ a postage of $n$ cents can be formed using just $5$-cent and $7$-cent stamps, where $n$ is any integer $\geq 24$. \\ \\
    is true.
}
\end{document}