\documentclass[12pt]{article}
\usepackage{amsmath,amsfonts,amssymb,amsthm} % Math packages
\usepackage[mathscr]{euscript}
\usepackage[utf8]{inputenc}
\usepackage[T1]{fontenc}
\usepackage{titling}
\usepackage[mathscr]{euscript}
\let\euscr\mathscr \let\mathscr\relax% just so we can load this and rsfs
\usepackage[scr]{rsfso}
\usepackage{pifont}
\usepackage{geometry} % Adjust page margins
\usepackage{titlesec} % Adjust section and subsection formatting
\geometry{a4paper, left=1in, right=1in, top=1in, bottom=1in}

\newcommand{\xlist}[1]{
    \begin{itemize}
        \renewcommand{\labelitemi}{$\centerdot$}
        #1
    \end{itemize}
    \newblock
    \\ \\
}

% Deductive Proofs %
\newcommand{\xsupposition}[1]{
    \underline{Suppositions}:
    \\ \\
    #1
    \\ \\
}
\newcommand{\xgoal}[1]{
    \underline{Goal}:
    \\ \\
    #1
    \\ \\
}
\newcommand{\xdeductions}{
    \underline{Deductions}:
    \\ \\
}
\newcommand{\xconclusion}[1]{
    \underline{Conclusion}:
    \\ \\
    #1
    \\ \\
}

% Inductive Proofs %
\newcommand{\xproof}[1]{
    \underline{Proof}:
    \\ \\
    #1
    \\ \\
}
\newcommand{\xbasisstep}{
    \underline{Basis Step}:
    \\ \\
}
\newcommand{\xinductivehypothesis}{
    \underline{Inductive Hypothesis}:
    \\ \\
}
\newcommand{\xinductivestep}{
    \underline{Inductive Steps}:
    \\ \\
}

\title{
  \textbf{CS-225: Discrete Structures in CS} \\
  Homework 5, Part 1
  }
\author{Noah Hinojos}
\date{\today}

\titleformat*{\subsection}{\normalsize\bfseries}

\begin{document}
\pagenumbering{gobble}
\maketitle


\section*{Exercise Set 5.2}


\subsection*{14}
\xproof{$P(n) \equiv \sum_{i=1}^{n+1}i \cdot 2^i = n \cdot 2^{n+2} + 2$ for every integer $n \geq 0$}
\xbasisstep
\begin{align*}
  P(0) = \sum_{i=1}^{0+1}i \cdot 2^i &= 0 \cdot 2^{0+2} + 2 \\
  \sum_{i=1}^{1}i \cdot 2^i &= 2 \\
  1\cdot2^1 &= 2 \\
  2 &= 2\\
\end{align*}
The base case $P(0)$ holds true.
\\ \\
\xinductivehypothesis
Suppose that for an aribitrary but particular integer $k$,
\begin{center}
  $P(k) \equiv \sum_{i=1}^{k+1}i \cdot 2^i = k \cdot 2^{k+2} + 2$ where $k \geq 0$
\end{center}
is true.
\\ \\
\xinductivestep
We must show that $P(k+1)$ is true. Hence we must demonstrate,
\begin{center}
  $P(k+1) = \sum_{i=1}^{k+2}i \cdot 2^i = (k+1) \cdot 2^{k+3} + 2$ where $k \geq 0$
\end{center}
\xlist{
  \item Let the expression $\sum_{i=1}^{k+2}i \cdot 2^i$ be referred to as the left-hand side (LHS) of the equation.
  \item Let the expression $(k+1) \cdot 2^{k+3} + 2$ be referred to as the right-hand side (RHS) of the equation.
  \item To show $P(k+1)$ is true, it must be shown that LHS is equal to RHS.

}
\\
$\sum_{i=1}^{k+2}i \cdot 2^i \Rightarrow (k+1) \cdot 2^{k+3} + 2$
\xlist{
  \item Recall the LHS is $\sum_{i=1}^{k+2}i \cdot 2^i$
  \item By algebra and the supposition of $P(k)$, the LHS can be simplified:
  \begin{align*}
    \sum_{i=1}^{k+2}i \cdot 2^i &= \sum_{i=1}^{k+2}i \cdot 2^i \\ 
    &= \sum_{i=1}^{k+1}i \cdot 2^i + (k+2)\cdot 2^{k+2} \\
    &= (k \cdot 2^{k+2} + 2) + (k\cdot 2^{k+2} + 2\cdot 2^{k+2}) \\
    &= 2(k\cdot 2^{k+2}) + 2^{k+3} + 2 \\
    &= k(2\cdot 2^{k+2}) + 2^{k+3} + 2 \\
    &= k(2^{k+3}) + 2^{k+3} + 2 \\
    &= (k+1) \cdot 2^{k+3} + 2
    \end{align*}
  \item Recall the RHS is $(k+1) \cdot 2^{k+3} + 2$
  \item Thus, LHS is equal to RHS
}
Thus, $P(k+1) = \sum_{i=1}^{k+2}i \cdot 2^i = (k+1) \cdot 2^{k+3} + 2$ where $k \geq 0$ was to be shown.
\\ \\
\xconclusion{
Since both the basis step and the inductive step have been proved, the original expression,
\begin{center}
  $P(n) \equiv \sum_{i=1}^{n+1}i \cdot 2^i = n \cdot 2^{n+2} + 2$ for every integer $n \geq 0$
\end{center}
must be true.
}


\subsection*{18}
\xproof{$P(n)=\prod_{i=2}^n\left(1-\frac{1}{i}\right) = \frac{1}{n}$ for every integer $n \geq 2$}
\xbasisstep
\begin{align*}
  P(2) = \prod_{i=2}^{2}\left(1-\frac{1}{i}\right) &= \frac{1}{2} \\
\left(1-\frac{1}{2}\right) &= \frac{1}{2} \\
  \frac{1}{2} &= \frac{1}{2} \\
\end{align*}
The base case $P(2)$ holds true.
\\ \\
\xinductivehypothesis
Suppose that for an aribitrary but particular integer $k$,
\begin{center}
  $P(k) \equiv \prod_{i=2}^{k}\left(1-\frac{1}{i}\right) = \frac{1}{k}$ where $k \geq 2$
\end{center}
is true.
\\ \\
\xinductivestep
We must show that $P(k+1)$ is true. Hence we must demonstrate,
\begin{center}
  $P(k+1) = \prod_{i=2}^{k+1}\left(1-\frac{1}{i}\right) = \frac{1}{k+1}$ where $k \geq 2$
\end{center}
\xlist{
  \item Let the expression $\prod_{i=2}^{k+1}\left(1-\frac{1}{i}\right)$ be referred to as the left-hand side (LHS) of the equation.
  \item Let the expression $\frac{1}{k+1}$ be referred to as the right-hand side (RHS) of the equation.
  \item To show $P(k+1)$ is true, it must be shown that LHS is equal to RHS. 
}
\\
$\prod_{i=2}^{k+1}\left(1-\frac{1}{i}\right) \Rightarrow \frac{1}{k+1}$
\xlist{
  \item Recall the LHS is $\prod_{i=2}^{k+1}\left(1-\frac{1}{i}\right)$
  \item By algebra and the supposition of $P(k)$, the LHS can be simplified:
  \begin{align*}
    \prod_{i=2}^{k+1}\left(1-\frac{1}{i}\right) &= \prod_{i=2}^{k}\left(1-\frac{1}{i}\right)\cdot\left(1-\frac{1}{k+1}\right) \\
    &= \left(\frac{1}{k}\right)\cdot\left(1-\frac{1}{k+1}\right) \\
    &= \frac{1}{k} - \frac{1}{k} \cdot \frac{1}{k+1} \\
    &= \frac{k+1}{k+1} \cdot \frac{1}{k} \ - \ \frac{1}{k} \cdot \frac{1}{k+1} \\
    & = \frac{k+1}{k^2 + k} - \frac{1}{k^2 + k} \\
    & = \frac{k}{k^2 + k} \\
    & = \frac{1}{k + 1} \\
  \end{align*}
  \item Recall the RHS is $\frac{1}{k+1}$
  \item Thus, LHS is equal to RHS.
}
Thus, $P(k+1) = \prod_{i=2}^{k+1}\left(1-\frac{1}{i}\right) = \frac{1}{k+1}$ where $k \geq 2$ was to be shown.
\\ \\
\xconclusion{
Since both the basis and inductive step have been proved, the original expression,
\begin{center}
  $P(n)=\prod_{i=2}^n\left(1-\frac{1}{i}\right) = \frac{1}{n}$ for every integer $n \geq 2$
\end{center}
must be true.
}


\section*{Exercise Set 5.3}

\subsection*{12}
\xproof{$P(n) \equiv 7^n - 2^n$ is divisible by 5, where $n \geq 0$}
\xbasisstep
\begin{align*}
  P(0) &= 7^0 - 2^0 \text{ is divisible by 5} \\
  &= 1-1 \text{ is divisible by 5} \\
  &= 0 \text{ is divisible by 5} \\
\end{align*}
The base case for $P(0)$ holds true.
\\ \\
\xinductivehypothesis
Suppose that for an arbitrary but particular integer $k$, such that $k \geq 0$,
\begin{center}
  $P(k) \equiv 7^k - 2^k \text{ is divisible by 5}$
\end{center}
is true. \\ \\
By definition of divisibility, this means that,
\begin{center}
  $P(k) \equiv 7^k - 2^k = 5r \text{ for some integer }r$
\end{center}
\xinductivestep
We must show that $P(k+1)$ is true. Hence we must demonstrate,
\begin{center}
  $P(k+1) = 7^{k+1} - 2^{k+1}$ is divisible by 5
\end{center}
\newblock
\\ \\
$P(k+1)$ is indeed divisible by 5.
\xlist{
  \item The expression for $P(k+1)$ can be simplified using algebra and the supposition of $P(k)$:
  \begin{align*}
    P(k+1) &= 7^{k+1} - 2^{k+1} \\
    &= 7\cdot7^k - 2\cdot2^k \\
    &= (5+2)\cdot7^k - 2\cdot2^k \\
    &= 5\cdot7^k - 2(7^k -2^k) \\
    &= 5\cdot7^k - 2(5r) \\
    &= 5(7^k - 2r) \\
  \end{align*}
  \item Hence, $P(k+1) = 5(7^k - 2r)$
  \item By closure, the internal expression $7^k - 2r$ is an integer because it is the product and summation of integers.
  \item Then by definition of divisibility, $5(7^k - 2r)$ is divisible by 5.
  \item By equallity, $P(k+1)$ must also be divisible by 5. 
  \item Hence, $P(k+1)$ is divisible by 5.
}
Thus, it was shown that $P(k+1) = 7^{k+1} - 2^{k+1}$ is divisible by 5.
\\ \\
\xconclusion{
Since both the basis and inductive step have been proved, the original expression,
\begin{center}
  $P(n) \equiv 7^n - 2^n$ is divisible by 5, where $n \geq 0$
\end{center}
must be true.}


\subsection*{15}
\xproof{$P(n) \equiv n(n^2+5)$ is divisible by 6, for each integer $n \geq 0$}
\xbasisstep
\begin{align*}
  P(0) &= 0\cdot(0^2+5) \text{ is divisible by 6} \\
  &= 0 \text{ is divisible by 6} \\
\end{align*}
The base case for $P(0)$ holds true.
\\ \\
\xinductivehypothesis
Suppose that for an arbitrary but particular integer $k$, such that $k \geq 0$,
\begin{center}
  $P(k) \equiv n(n^2+5) \text{ is divisible by 6}$
\end{center}
is true. \\ \\
By definition of divisibility, this means that,
\begin{center}
  $P(k) \equiv n(n^2+5) = 6r \text{ for some integer }r$
\end{center}
\xinductivestep
We must show that $P(k+1)$ is true. Hence we must demonstrate,
\begin{align*}
  P(k+1) = (n+1)\cdot((n+1)^2 + 5) \text{ is divisible by 6} \\
\end{align*}
\newblock
\\
$p(p+1)$ can be expressed as $2q$ where $p$ and $q$ are some integers.
\xlist{
  \item Let $p$ and $q$ be integers.
  \item Let $t = p(p+1)$.
  \item By closure, $t$ is an integer because it is the product and summation of integers.
  \item By the Parity Property (Theorem 4.5.2), $p$ and $p+1$ have opposite parity; one expression is even and the other is odd.
  \item By definition of even, $t$ is even because it is the product of an even and an odd integer.
  \item Hence, $t$ can be expressed as $2q$.
  \item By equality, $p(p+1) = t = 2q$.
  \item Therefore, $p(p+1)$ can be expressed as $2q$ where $p$ and $q$ are some integers.
}
\\ \\
$P(k+1)$ is indeed divisible by 6.
\xlist{
  \item The expression for $P(k+1)$ can be simplified using algebra and the supposition of $P(k)$:
  \begin{align*}
    P(k+1) &= (n+1)\cdot((n+1)^2 + 5) \\
    &= (n+1)\cdot(n^2+5+n+1) \\
    &= (n+1)\cdot((n^2+5)+(2n+1)) \\
    &= n(n^2+5)+(n^2+5)+(n+1)(2n+1)\\
    &= 6r +3n^2+3n+6\\
    &= 6(r+1)+3n^2+3n\\
    &= 6(r+1)+3(n(n+1))\\
  \end{align*}
  \item Let $m$ be some integer.
  \item The expression $n(n+1) = 2m$ as demonstrated by the first deduction.
  \item This can be subsituted into our new expression for $P(k+1)$. Contiuing the algebra from before:
  \begin{align*}
    P(k+1) &= 6(r+1)+3(n(n+1))\\
    &= 6(r+1)+3(2m)\\
    &= 6(r+1)+6m\\
    &= 6(r+1+m)\\
  \end{align*}
  \item Hence, $P(k+1) = 6(r+1+m)$
  \item By closure, the internal expression $r+1+m$ is an integer because it is the summation of integers. 
  \item Then by definition of divisibility, $6(r+1+m)$ is divisible by 6.
  \item By equality, $P(k+1)$ must also be divisible by 6.
  \item Hence, $P(k+1)$ is divisible by 6.
}
Thus, it was shown that $P(k+1)$ is divisible by 6.
\\ \\
\xconclusion{
  Since both the basis and inductive step have been proved, the original expression,
  \begin{center}
    $P(n) \equiv n(n^2+5)$ is divisible by 6, for each integer $n \geq 0$
  \end{center}
  must be true.
}
\subsection*{23 - b}
\xproof{$P(n) \equiv n! > n^2$, for each integer $n\geq4$}
\xbasisstep
\begin{align*}
  P(4) &= 4! > 4^2 \\
  &= 24 > 16 \\
\end{align*}
The base case for $P(4)$ holds true.
\\ \\
\xinductivehypothesis
Suppose that for an arbitrary but particular integer $k$, such that $k \geq 4$,
\begin{center}
  $P(k) \equiv k! > k^2$
\end{center}
is true. \\ \\
\xinductivestep
We must show that $P(k+1)$ is true. Hence we must demonstrate,
\begin{align*}
  P(k+1) = (k+1)! > (k+1)^2 \\
\end{align*}
\xlist{
  \item Let $(k+1)!$ be referred to as the left-hand side (LHS) of the equation.
  \item Let $(k+1)^2$ be referred to as the right-hand side (RHS) of the equation.
  \item To show $P(k+1)$ is true, it must be shown that LHS is greater than RHS.
}
\\ \\
$P(k+1)$ is indeed true.
\xlist{
  \item The expression for $P(k+1)$ can be simplified using algebra:
  \begin{align*}
    P(k+1) = (k+1)! &> (k+1)^2 \\
    (k+1)k! &> (k+1)^2 \\
    k! &> (k+1) \\
  \end{align*}
  \item By Proof\#2 in the provided Additional Proofs, it is true that $k! > (k+1)$ for every integer $k \geq 4$. 
  \item Hence, $P(k+1)$ is indeed true.
}
Thus, it was shown that $P(k+1)$ is true.
\\ \\
\xconclusion{
  Since both the basis and inductive step have been proved, the original expression,
  \begin{center}
    $P(n) \equiv n! > n^2$, for each integer $n \geq 4$
  \end{center}
  must be true.}
\end{document}