\documentclass[12pt]{article}
\usepackage{amsmath,amsfonts,amssymb,amsthm} % Math packages
\usepackage[mathscr]{euscript}
\usepackage[utf8]{inputenc}
\usepackage[T1]{fontenc}
\usepackage{titling}
\usepackage[mathscr]{euscript}
\let\euscr\mathscr \let\mathscr\relax% just so we can load this and rsfs
\usepackage[scr]{rsfso}
\usepackage{pifont}
\usepackage{geometry} % Adjust page margins
\usepackage{titlesec} % Adjust section and subsection formatting
\geometry{a4paper, left=1in, right=1in, top=1in, bottom=1in}

\newcommand{\xlist}[1]{
    \begin{itemize}
        \renewcommand{\labelitemi}{$\centerdot$}
        #1
    \end{itemize}
    \newblock
    \\ \\
}

\newcommand{\xsupposition}[1]{
    \underline{Suppositions}:
    \\ \\
    #1
    \\ \\
}

\newcommand{\xgoal}[1]{
    \underline{Goal}:
    \\ \\
    #1
    \\ \\
}

\newcommand{\xdeduction}{
    \underline{Deductions}:
    \\ \\
}

\newcommand{\xconclusion}[1]{
    \underline{Conclusion}:
    \\ \\
    #1
    \\ \\
}

\newcommand{\xproof}{
    \underline{Proof}:
    \\ \\
}

\newcommand{\xbasistep}{
    \underline{Basis Step}:
    \\ \\
}

\newcommand{\xinductivehypothesis}{
    \underline{Inductive Hypothesis}:
    \\ \\
}

\newcommand{\xinductivesteps}{
    \underline{Inductive Steps}:
    \\ \\
}

\title{
  \textbf{CS-225: Discrete Structures in CS} \\
  Homework 6
  }
\author{Noah Hinojos}
\date{\today}

\titleformat*{\subsection}{\normalsize\bfseries}

\begin{document}
\maketitle
\section*{1}
\underline{Given}:
\\ \\
$e_k = 5e_{k-1} + 3 \text{ for all integers } k \geq 2$ \\
$e_1 = 2$
\\ \\
\underline{Iterations}:
\\ \\
\begin{align*}
  e_1 &= 2 \\
  e_2 &= 5e_1 + 3 \\
  e_3 &= 5e_2 + 3 \\
  &= 5(5e_1 + 3) + 3 \\
  e_4 &= 5e_3 + 3\\
  &= 5(5(5e_1 + 3) + 3) + 3\\
  &= 5^3\cdot e_1 + 5^2\cdot3 + 5\cdot 3 + 3 \\
  &= 5^{3}\cdot 2 + 3\sum_{i=0}^{3}5^{i-1} \\
  &= 5^{4-1}\cdot 2 + 3\sum_{i=0}^{4-2}5^{i} \\
  e_n &= 5^{n-1}\cdot 2 + 3\sum_{i=0}^{n-2}5^{i} \\
\end{align*}
\\ \\
\underline{Guess}:
\\ \\
$e_n = 5^{n-1}\cdot 2 + 3\sum_{i=0}^{n-2}5^{i} \text{ for every integer }n \geq 1$


\newpage
\section*{2}
\xproof
Given the sequence $e_1, e_2, \ldots, e_n$
\begin{align*}
  e_1 &= 2 \hspace*{13cm}\\
  e_k &= 5e_{k-1} + 3 \text{ for all integers } k \geq 2
\end{align*}
Let $P(n) \equiv e_n = 5^{n-1}\cdot 2 + 3\sum_{i=0}^{n-2}5^{i} \text{ for every integer }n \geq 1$
\\ \\
\xbasistep
\begin{align*}
  P(1) = e_1 &= 5^{1-1}\cdot 2 + 3\sum_{i=0}^{1-2}5^{i} \\
  &= 5^0 \cdot 2 + 3 \sum _{i=0}^{-1}5^{i} \\
  &= 2 + 3\cdot 0 \\
  &= 2 \\
\end{align*}
\newblock
\\ \\
The base case for $P(1)$ holds true.
\\ \\
\xinductivehypothesis
Suppose that for an arbitrary but particular integer $j$, \\
$$ P(j) \equiv e_j = 5^{j-1}\cdot 2 + 3\sum_{i=0}^{j-2}5^{i} \text{ for every integer }j \geq 1$$
is true.
\\ \\
\xinductivesteps
We must show that $P(j+1)$ is true. Hence we must demonstrate that,
$$ P(j+1) \equiv e_{j+1} = 5^{j}\cdot 2 + 3\sum_{i=0}^{j-1}5^{i} $$
\\
By algebra, $j+1$ can be plugged into the original sequence defintion:
\begin{align*}
  P(j+1) = e_{j+1} &= 5e_j+3 \\
  &= 5\left(5^{j-1}\cdot 2 + 3\sum_{i=0}^{j-2}5^{i}\right) + 3  \text{ \hspace*{1cm}(Subsitution of Inductive Hypothesis)}\\
  &= 5\cdot5^{j-1}\cdot 2 + 5\cdot 3\sum_{i=0}^{j-2}5^{i} + 3\sum_{i=0}^{0}5^i \\ 
  &= 5^j\cdot 2 + 3\sum_{i=0}^{j-2}5^{i+1} + 3\sum_{i=0}^{0}5^i \\ 
  &= 5^j\cdot 2 + 3\sum_{i=1}^{j-1}5^{i} + 3\sum_{i=0}^{0}5^i \\ 
  &= 5^j\cdot 2 + 3\left(\sum_{i=1}^{j-1}5^{i} + \sum_{i=0}^{0}5^i\right) \\ 
  &= 5^j\cdot 2 + 3\sum_{i=0}^{j-1}5^{i} \\ 
\end{align*}
\newblock
Thus, with was shown that $P(j+1) \equiv e_{j+1} = 5^{j}\cdot 2 + 3\sum_{i=0}^{j-1}5^{i}$ holds true.
\\ \\
\xconclusion{
Since both the basis and inductive step have been proved, the original expression, 
$$P(n) \equiv e_n = 5^{n-1}\cdot 2 + 3\sum_{i=0}^{n-2}5^{i} \text{ for every integer }n \geq 1$$
is true.
}


\newpage
\section*{3}
\underline{Given}:
\\ \\
$t_k = t_{k-1} + 7k^3 + 4k + 5 \text{ for all integers } k \geq 1$ \\
$t_0 = 2$
\\ \\
\underline{Iterations}:
\\ \\
\begin{align*}
  t_0 &= 2 \\
  t_1 &= t_0 + 7\cdot 1^3 + 4\cdot1 + 5 \\
  &= 2 + 7\cdot 1^3 + 4\cdot1 + 5 \\
  t_2 &= t_1 + 7\cdot 2^3 + 4\cdot2 + 5 \\
  &= (2 + 7\cdot 1^3 + 4\cdot1 + 5) + 7\cdot 2^3 + 4\cdot2 + 5 \\
  &= 2 + 7(1^3 + 2^3) + 4(1+2) + 2(5) \\
  t_3 &= t_2 + 7\cdot 3^3 + 4\cdot2 + 5 \\
  &= ((2 + 7\cdot 1^3 + 4\cdot1 + 5) + 7\cdot 2^3 + 4\cdot2 + 5) + 7\cdot 3^3 + 4\cdot2 + 5 \\
  &= 2 + 7(1^3 + 2^3 + 3^3) + 4(1+2+3) + 3(5) \\
  t_n &= 7\sum_{i=1}^{n}n^3 + 4\sum_{i=1}^{n}n + 5n + 2
\end{align*}
\\ \\
\underline{Guess}:
\\ \\
$t_n = 7\sum_{i=1}^{n}n^3 + 4\sum_{i=1}^{n}n + 5n + 2 \text{ for every integer }n \geq 0$ \\
\newpage
\section*{4}
I - Base: The strings $ab$ and $ba$ are in set $S$.
\\ \\
II - Recursion: New strings are formed according to the following rules:
\begin{itemize}
  \item [a.] If $u$ is any string in $S$, or if $u$ is either the individual character $a$ or $b$; then
  \begin{center}
    $aub$ is a string in $S$\\
  \end{center}
  where $aub$ is the concatenation of $a$, $u$ and $b$; $aub$ is obtained by appending $a$ on the left of $u$, and $b$ on the right of $u$. 
  \item [b.] If $u$ is any string in $S$, or if $u$ is either the individual character $a$ or $b$; then
  \begin{center}
    $bua$ is a string in $S$\\
  \end{center}
  where $bua$ is the concatenation of $b$, $u$ and $a$; $bua$ is obtained by appending $b$ on the left of $u$, and $a$ on the right of $u$.
\end{itemize} 
III - Restriction: Every string in $S$ is obtainable from the base and the recursion.


\newpage
\section*{5}
I - Base: The individual string characters $a$ and $b$ are in set $S$.
\\ \\
II - Recursion: New strings are formed according to the following rules:
\begin{itemize}
  \item [a.] If $u$ is any string in $S$, then
  \begin{center}
    $aua$ is a string in $S$\\
  \end{center}
  where $aua$ is the concatenation of $a$, $u$ and $b$; $aua$ is obtained by appending $a$ on both the left and right side of $u$.
  \item [b.] If $u$ is any string in $S$, then
  \begin{center}
    $bub$ is a string in $S$\\
  \end{center}
  where $bub$ is the concatenation of $a$, $u$ and $b$; $bub$ is obtained by appending $b$ on both the left and right side of $u$.
  \item [c.] If $u$ is any string in $S$, then
  \begin{center}
    $aub$ is a string in $S$\\
  \end{center}
  where $aub$ is the concatenation of $a$, $u$ and $b$; $aub$ is obtained by appending $a$ on the left of $u$, and $b$ on the right of $u$.
  \item [d.] If $u$ is any string in $S$, then
  \begin{center}
    $bua$ is a string in $S$\\
  \end{center}
  where $bua$ is the concatenation of $a$, $u$ and $b$; $bua$ is obtained by appending $b$ on the left of $u$, and $a$ on the right of $u$. 
\end{itemize}
III - Restriction: Every string in $S$ is obtainable from the base and the recursion.
\end{document}