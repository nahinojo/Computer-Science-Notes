\documentclass[12pt]{article}
\usepackage{amsmath,amsfonts,amssymb,amsthm} % Math packages
\usepackage[mathscr]{euscript}
\usepackage[utf8]{inputenc}
\usepackage[T1]{fontenc}
\usepackage{titling}
\usepackage[mathscr]{euscript}
\let\euscr\mathscr \let\mathscr\relax% just so we can load this and rsfs
\usepackage[scr]{rsfso}
\usepackage{pifont}
\usepackage{geometry} % Adjust page margins
\usepackage{titlesec} % Adjust section and subsection formatting
\geometry{a4paper, left=1in, right=1in, top=1in, bottom=1in}

\newcommand{\xlist}[1]{
    \begin{itemize}
        \renewcommand{\labelitemi}{$\centerdot$}
        #1
    \end{itemize}
    \newblock
}

\newcommand{\xsupposition}[1]{
    \underline{Suppositions}:
    \\ \\
    #1
    \\ \\
}

\newcommand{\xgoal}[1]{
    \underline{Goal}:
    \\ \\
    #1
    \\ \\
}

\newcommand{\xdeduction}{
    \underline{Deductions}:
    \\ \\
}

\newcommand{\xconclusion}[1]{
    \underline{Conclusion}:
    \\ \\
    #1
    \\ \\
}

\newcommand{\xproof}{
    \underline{Proof}:
    \\ \\
}

\newcommand{\xbasistep}{
    \underline{Basis Step}:
    \\ \\
}

\newcommand{\xinductivehypothesis}{
    \underline{Inductive Hypothesis}:
    \\ \\
}

\newcommand{\xinductivesteps}{
    \underline{Inductive Steps}:
    \\ \\
}

\title{
  \textbf{CS-225: Discrete Structures in CS} \\
  Week 7 and 8 Quiz
  }
\author{Noah Hinojos}
\date{\today}

\titleformat*{\subsection}{\normalsize\bfseries}

\begin{document}
\maketitle
\section*{Responses}
\subsection*{1 - a}
Let the pigeonholes represent the even integers within the range from 0 to 50. 
By definition of even, there are 26 even integers within this range. 
Hence, there are 26 pigeonholes.
\\ \\
The pigeons would then represent the selected integers. 
Hence, the pigeonholes can support a maximum of 26 pigeons.
Or rather, you can select up to 26 even integers.
\\ \\
Then to guarantee that a selected integer is odd, you first must gaurantee your selection fulfills the maximum number of even pigeonholes, then add one extra.
Therefore, you must select at least 27 integers within the range of 0 to 50 to gaurantee one of the selected integers is odd. 
\subsection*{1 - b}
Let the pigeonholes represent the color options. Hence there are 2 pigeonholes, one for red and one for blue.
\\ \\
Let the pigeons represent the selected balls themselves. A selected ball (pigeon) may correspond to either the color red or blue (pigeonhole).
\\ \\
To fullfill the requirement that a selection contains a pair of the same color, you essentially must have multiple pigeons in the same hole. 
To guarantee this, the selection of pigeons must be greater than the number of pigeonholes. Hence, the selection must then be 3 or more balls.
\\ \\
To gaurantee a selection of balls contain a pair of the same color, you must select at least 3 balls.
\subsection*{2 - a}
Let $T$ represent the set of all students. Hence, $N(T) = 50$. \\
Let $A$ represent the subset of all students that like coffee. Hence, $N(A) = 23$. \\ 
Let $B$ represent the subset of all students that like tea. Hence, $N(B) = 36$. \\
Furthermore, there are 12 students that like both. Hence, $N(A \cap B) = 12$.
\\ \\
The number of students that like coffee or tea can then be calculated: \\
\begin{align*}
  N(A \cup B) &= N(A) + N(B) - N(A \cap B) \\
  &= 23 + 36 - 12 \\ 
  &= 47 \\
\end{align*}
The number of students that like neither coffee nor tea can then be calculated: \\
\begin{align*}
  A^c \cap B^c &= (A \cup B)^c \\
  &= T - (A \cup B)\\
  N(A^c \cap B^c) &= N(T - (A \cup B)) \\
  &= N(T) - N(A \cup B) \\
  &= 50 - 47 \\
  &= 3
\end{align*}
There are 3 students that like neither coffee nor tea.
\subsection*{2 - b}
\begin{itemize}
  \item [i.] Considering the restriction, there are 3 locations for any 26 letters as well as 2 locations for any ten digits.
  \\ \\
  Hence, there are $26^3 \cdot 10^2$ license plates where the plates begin with A and end in 0.
  \item [ii.] Let $T$ represent all possible license plates with no \textit{specific} character restrictions. \\
  Let $E_{99}$ represent the set of license plates that end in 99. \\
  Let $E_{\sim99}$ represent the set of license plates that do not end in 99.
  \\ \\
  The number of license plates that do not end in 99 can then be calculated: \\
  \begin{align*}
    N(E_{\sim99}) &= N(T) - N(E_{99}) \\
    &= 26^4 \cdot 10^3 - 26^4 \cdot 10^1 \\
    &= 452406240 \\
  \end{align*}
  There are 452,406,240 license plates that do not end in 99.
\end{itemize}
\subsection*{3 - a}
$10!$
\subsection*{3 - b}
$2\cdot 9!$
\subsection*{3 - c}
Let $T$ represent the set of all combinations. \\
Let $E$ represent the subset of all combinations where $P$, $O$, and $R$ are together. 
\begin{align*}
  N(T) &= 10! \\
  N(E) &= 3! \cdot 8! \\
  N(E^c) &= 10! - 3! \cdot 8! \\
\end{align*}
\subsection*{4 - a}
$$\binom{6}{4} + \binom{9}{4}$$
\subsection*{4 - b}
$$\binom{9}{2}\cdot\binom{6}{2}$$
\subsection*{5 - a}
$$P(6, 3)$$
\subsection*{5 - b}
$$\frac{P(6, 3)}{2}$$
\subsection*{6}
$$\binom{8}{2}\binom{6}{2}\binom{4}{2}\binom{2}{1}\binom{1}{1}$$
\subsection*{7}
$$\binom{21 + 4 - 1}{21}$$
\subsection*{8 - a}
Let $T$ represent the set of all pastry combinations. \\
Let $E_{\geq3}$ represent the subset of all combinations where there are at least 3 eclairs. \\
\begin{align*}
  N(E_{\geq3}) &= \binom{27+5-1}{27}
\end{align*}
\subsection*{8 - b}
Let $E_{\leq5}$ represent the subset of all combinations where there are at most 5 eclairs. \\
\begin{align*}
  N(E_{\leq5}) &= N(T) - N(E_{\geq6})\\
  &= \binom{30+5-1}{30} - \binom{24+5-1}{24}
\end{align*}
\end{document}
