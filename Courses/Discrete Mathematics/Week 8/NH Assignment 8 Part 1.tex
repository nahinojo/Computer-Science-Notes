\documentclass[12pt]{article}
\usepackage{amsmath,amsfonts,amssymb,amsthm} % Math packages
\usepackage[mathscr]{euscript}
\usepackage[utf8]{inputenc}
\usepackage[T1]{fontenc}
\usepackage{titling}
\usepackage[mathscr]{euscript}
\let\euscr\mathscr \let\mathscr\relax% just so we can load this and rsfs
\usepackage[scr]{rsfso}
\usepackage{pifont}
\usepackage{geometry} % Adjust page margins
\usepackage{titlesec} % Adjust section and subsection formatting
\geometry{a4paper, left=1in, right=1in, top=1in, bottom=1in}

\newcommand{\xlist}[1]{
    \begin{itemize}
        \renewcommand{\labelitemi}{$\centerdot$}
        #1
    \end{itemize}
    \newblock
}

\newcommand{\xsupposition}[1]{
    \underline{Suppositions}:
    \\ \\
    #1
    \\ \\
}

\newcommand{\xgoal}[1]{
    \underline{Goal}:
    \\ \\
    #1
    \\ \\
}

\newcommand{\xdeduction}{
    \underline{Deductions}:
    \\ \\
}

\newcommand{\xconclusion}[1]{
    \underline{Conclusion}:
    \\ \\
    #1
    \\ \\
}

\newcommand{\xproof}{
    \underline{Proof}:
    \\ \\
}

\newcommand{\xbasistep}{
    \underline{Basis Step}:
    \\ \\
}

\newcommand{\xinductivehypothesis}{
    \underline{Inductive Hypothesis}:
    \\ \\
}

\newcommand{\xinductivesteps}{
    \underline{Inductive Steps}:
    \\ \\
}

\title{
  \textbf{CS-225: Discrete Structures in CS} \\
  Assignment 8 Part 1
  }
\author{Noah Hinojos}
\date{\today}

\titleformat*{\subsection}{\normalsize\bfseries}

\begin{document}
\maketitle
\section*{Exercise Set 9.2}
\subsection*{32 - c}
There are seven units that can be arranged, the phrase $GOR$ and the other six letters, hence:
\\ \\
$7! = 5040$
\\ \\
There are 5040 ways to arrange $ALGORITHM$ where $GOR$ is kept together as a single unit. 
\subsection*{39 - d}
With the first 3 characters $OR$ at the beginning, there are now 7 possible letters to choose in the 4 remaining locations:
\\ \\
\begin{align*}
  P(7, 4) &= \frac{7!}{(7-4)!} \\
  &= \frac{7!}{3!} \\
  &= 7 \cdot 6 \cdot 5 \cdot 4 \\  
  &= 840
\end{align*}
\\ \\
There are 840 ways to arrange six characters of $ALGORITHM$ where $OR$ is kept together as a single unit at the beginning of every phrase.
\section*{Exercise Set 9.5}
\subsection*{20 - b}
Essentially, we're calculating the number of distinguishable ways to order $ILLIMICRO$.
\xlist{
  \item For $I$, there are $\binom{9}{3}$ subsets.
  \item For $L$, there are $\binom{6}{2}$ subsets.
  \item For all indivisual unique characters, there are $4!$ subsets.
}
$$\binom{9}{3}\binom{6}{2}\cdot4! = 84\cdot15\cdot4! =30240$$
\\
There are 30240 disntinguishable orderings of the letters $MILLICRON$ where the ordering begins with $M$ and ends with $N$.
\subsection*{20 - c}
\begin{align*}
  \frac{9!}{3!2!2!1!} = 15120
\end{align*}
There are 30240 disntinguishable orderings of the letters $MILLICRON$ where the characters $CR$ and $ON$ are each next to each other respectively.
\section*{Canvas Problems}
\subsection*{1}
Let's begin by ignoring the roundness of the table for a moment. 
We have 5 men and 7 women across 12 seats, and no two men may be together.
\\ \\
Starting 7 women, they \textit{themselves} can be ordered $7!$ different ways.
\\ \\
Now for the 5 men, we have to place them in between the 7 women. 
There are 7 different locations between women to place these men.
The number of ways to distribute the 5 men across 7 locations is therefore $P(7, 5)$.
\\ \\
To account for the roundness of the table, consider the number of indistinguishable copies each permutation has. 
By shifting everyone to the right by one, we create arrive copy.
Therefore, every permutation must have 12 total copies of itself for the 12 possible rotational shifts. 
Divide the total number of permutations thus far by 12 in order to account for this and ensure unique seating.
\\ \\
The number of ways to distribute the 5 men and 7 women across a round table where no two men sit together:
\\
$$7! \cdot \frac{7!}{(7-5)!} \cdot \frac{1}{12} $$ 

\subsection*{2}
Number of ways to arrange 7 characters of $PALINDROME$ where the first two letters are $PA$:
\\
$$P(8, 5) = \frac{8!}{(8-5)!}$$
\\
Number of ways to arrange 7 characters of $PALINDROME$ where the last two letters are $ME$:
\\
$$P(8, 5) = \frac{8!}{(8-5)!}$$
\\
Total number of ways:
\\
$$\frac{8!}{(8-5)!} + \frac{8!}{(8-5)!} = \frac{2\cdot8!}{(8-5)!}$$
\subsection*{3}
The total number of ways to arrange $A,B,C,D,E$ is $5! = 120$.
\\ \\
In every case, either $E$ is before $A$ or $E$ is after $A$. Due to symmetry, there are an equal number of cases of each scenario.
Therefore, the number of ways to arrange $A,B,C,D,E$ where $E$ is after $A$ is: 
\\
$$\frac{5!}{2} = 60$$.
\end{document}