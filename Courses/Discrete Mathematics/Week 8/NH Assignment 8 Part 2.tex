\documentclass[12pt]{article}
\usepackage{amsmath,amsfonts,amssymb,amsthm} % Math packages
\usepackage[mathscr]{euscript}
\usepackage[utf8]{inputenc}
\usepackage[T1]{fontenc}
\usepackage{titling}
\usepackage[mathscr]{euscript}
\let\euscr\mathscr \let\mathscr\relax% just so we can load this and rsfs
\usepackage[scr]{rsfso}
\usepackage{pifont}
\usepackage{geometry} % Adjust page margins
\usepackage{titlesec} % Adjust section and subsection formatting
\geometry{a4paper, left=1in, right=1in, top=1in, bottom=1in}

\newcommand{\xlist}[1]{
    \begin{itemize}
        \renewcommand{\labelitemi}{$\centerdot$}
        #1
    \end{itemize}
    \newblock
}

\newcommand{\xsupposition}[1]{
    \underline{Suppositions}:
    \\ \\
    #1
    \\ \\
}

\newcommand{\xgoal}[1]{
    \underline{Goal}:
    \\ \\
    #1
    \\ \\
}

\newcommand{\xdeduction}{
    \underline{Deductions}:
    \\ \\
}

\newcommand{\xconclusion}[1]{
    \underline{Conclusion}:
    \\ \\
    #1
    \\ \\
}

\newcommand{\xproof}{
    \underline{Proof}:
    \\ \\
}

\newcommand{\xbasistep}{
    \underline{Basis Step}:
    \\ \\
}

\newcommand{\xinductivehypothesis}{
    \underline{Inductive Hypothesis}:
    \\ \\
}

\newcommand{\xinductivesteps}{
    \underline{Inductive Steps}:
    \\ \\
}

\title{
  \textbf{CS-225: Discrete Structures in CS} \\
  Assignment 8 Part 1
  }
\author{Noah Hinojos}
\date{\today}

\titleformat*{\subsection}{\normalsize\bfseries}

\begin{document}
\maketitle
\section*{Exercise Set 9.5}
\subsection*{8 - b}
\begin{itemize}
  \item [i.] $$\binom{6}{4}\binom{8}{6} = 420$$
  \item [ii.] All possible choices of 10 must contain at least one: $$\binom{14}{10} = 1001$$
  \item [iii.] $$\binom{6}{2}\binom{8}{8} + \binom{6}{3}\binom{8}{8}$$ 
\end{itemize}
\subsection*{8 - c}
Q1 only: $$ Q_1 = \binom{12}{9}$$
Q2 only: $$ Q_2 = \binom{12}{9}$$
Nether Q1 or Q2: $$ Q_{neither} = \binom{12}{10}$$
\underline{Total}:$$Q1 + Q2 + Q_{neither} = \binom{12}{9} + \binom{12}{9} + \binom{12}{10}$$
\subsection*{12}
An even sum occurs when either two even or two odd numbers are summated. 
There are 50 even numbers and 51 odd numbers in this list.
\\ \\
Summation of two even numbers: $$\binom{50}{2}$$
\\ \\
Summation of two odd numbers: $$\binom{51}{2}$$
\\ \\
\underline{Total}: $$\binom{50}{2} + \binom{51}{2}$$
\subsection*{17 - c}
$$\binom{10}{3}$$
\subsection*{17 - d}
$$\binom{9}{3}$$
\section*{Exercise Set 9.6}
\subsection*{14}
Every value must be greater than or equal to 10. Hence, we're distributing 450 elementsacross 5 locations:
$$\binom{450+5-1}{450}$$
\subsection*{19 - a}
Let $T$ represent the set of all possible selections of 20 pastries where there are 6 pastry types. \\
Let $E_i$ represent the set of all selections within $T$ where each selection contains $i$ eclairs. 
\\ \\
Then the total number of ways to select 20 of 6 types of pastries:
$$N(T) = \binom{20+6-1}{20}$$
Number of ways to select 20 of 6 types of pastries where 11 or more are eclairs.
This is equivocal to selecting 9 of 6 types of pastries:
$$N(E_{\geq11}) = \binom{9+6-1}{9}$$
\underline{Answer}: Now we can calculate the number of ways to select 20 of 6 types of pastries where at most 10 are eclairs:
\begin{align*}
  E_{\leq10} &= T \cap E_{\geq11}^c \\
  \\
  &= T - E_{\geq11} \\
  \\
  N(E_{\leq10}) &= N(T - E_{\geq11}) \\
  \\
  &= N(T) - N(E_{\geq11})\\
  \\
  &= \binom{20+6-1}{20} - \binom{9+6-1}{9}
\end{align*}
\subsection*{19 - b}
Let $S_i$ represent the set of all possible selections of pastries where each selection contains  $i$ napolean slices.
\\ \\
Next, let's calculate the number of ways to select 20 of 6 types of pastries where 9 or more are napolean slices.
This is equivocal to selecting 11 of 6 types of pastries: \\
\begin{align*}
  N(S_{\geq9}) &= \binom{11+6-1}{11} \\
\end{align*}
\underline{Answer}: Number of ways to select 20 of 6 types of pastries where at most 8 are napolean slices and at most 10 are eclairs.

\begin{align*}
  E_{\leq10} \cap S_{\leq8} &= E_{\leq10} - S_{\geq9}  \\
  \\
  N(E_{\leq10} - S_{\leq9}) &= N(E_{\leq10}) - N(S_{\geq9}) \\
  \\
  &= \binom{20+6-1}{20} - \binom{9+6-1}{9} - \binom{11+6-1}{11}
\end{align*}
\subsection*{20 - a}
Let $T$ represent the set of all possible selections of 30 batteries where there are 8 battery types. \\
Let $A_i$ represent the set of all selectrions within $T$ where each selection contains $i$ A76 batteries. \\
The total number of ways to select 30 of 8 types of batteries:
$$N(T) = \binom{30+8-1}{30}$$
Now we can calculate the number of ways to select 30 of 8 types of batteries where at 11 or more are A76 batteries.
This is equivocal to selecting 19 of 8 types of batteries:
$$N(A_{\geq11}) = \binom{19+8-1}{19}$$
\underline{Answer}: Finally, let's calculate the number of ways to select 30 of 8 types of batteries where at most 10 are A76 batteries:
\begin{align*}
  A_{\leq10} &= T \cap A_{\geq11}^c \\
  \\ 
  &= T - A_{\geq11} \\
  \\
  N(A_{\leq10}) &= N(T - A_{\geq11}) \\
  \\
  &= N(T) - N(A_{\geq11})\\
  \\
  &= \binom{30+8-1}{30} - \binom{19+8-1}{19}
\end{align*}
\subsection*{20 - b}
Let $D_i$ represent the set of all possible selections of batteries where there are $i$ D303 batteries.
\\ \\
Number of ways to select 30 of 8 types of batteries where 7 or more are D303 batteries.
This is equivocal to selecting 23 of 8 types of batteries: \\
\begin{align*}
  N(D_{\geq7}) &= \binom{23+8-1}{23} \\
\end{align*}
\underline{Answer}: Number of ways to select 30 of 8 types of batteries where at most 6 are D303 batteries and at most 10 are A76 batteries.

\begin{align*}
  A_{\leq10} \cap D_{\leq6} &= A_{\leq10} - D_{\geq7}  \\
  \\
  N(A_{\leq10} - D_{\leq6}) &= N(A_{\leq10}) - N(D_{\geq7}) \\
  \\
  &= \binom{30+8-1}{30} - \binom{19+8-1}{19} - \binom{23+8-1}{23}
\end{align*}


\end{document}
