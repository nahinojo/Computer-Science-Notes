\documentclass[12pt]{article}
\usepackage{amsmath,amsfonts,amssymb,amsthm} % Math packages
\usepackage[mathscr]{euscript}
\usepackage[utf8]{inputenc}
\usepackage[T1]{fontenc}
\usepackage{titling}
\usepackage[mathscr]{euscript}
\let\euscr\mathscr \let\mathscr\relax% just so we can load this and rsfs
\usepackage[scr]{rsfso}
\usepackage{pifont}
\usepackage{geometry} % Adjust page margins
\usepackage{titlesec} % Adjust section and subsection formatting
\geometry{a4paper, left=1in, right=1in, top=1in, bottom=1in}

\newcommand{\xlist}[1]{
    \begin{itemize}
        \renewcommand{\labelitemi}{$\centerdot$}
        #1
    \end{itemize}
    \newblock
    \\ \\
}

\newcommand{\xsupposition}[1]{
    \underline{Suppositions}:
    \\ \\
    #1
    \\ \\
}

\newcommand{\xgoal}[1]{
    \underline{Goal}:
    \\ \\
    #1
    \\ \\
}

\newcommand{\xdeduction}{
    \underline{Deductions}:
    \\ \\
}

\newcommand{\xconclusion}[1]{
    \underline{Conclusion}:
    \\ \\
    #1
    \\ \\
}

\newcommand{\xproof}{
    \underline{Proof}:
    \\ \\
}

\newcommand{\xbasistep}{
    \underline{Basis Step}:
    \\ \\
}

\newcommand{\xinductivehypothesis}{
    \underline{Inductive Hypothesis}:
    \\ \\
}

\newcommand{\xinductivesteps}{
    \underline{Inductive Steps}:
    \\ \\
}

\title{
  \textbf{CS-225: Discrete Structures in CS} \\
  Assignment 7 Part 1
  }
\author{Noah Hinojos}
\date{\today}

\titleformat*{\subsection}{\normalsize\bfseries}

\begin{document}
\maketitle
\section*{Exercise Set 9.2}
\subsection*{11 - c}
$1 \cdot 2^6 \cdot 1 = 64$


\subsection*{17 - d}
$5\cdot8\cdot8\cdot7 = 2240$

\subsection*{18 - c}
$10\cdot9\cdot8\cdot7 = 5040$

\subsection*{28}
\begin{align*}
  n_i &= b-a+1 \\
  n_j &= d-c+1 \\
  n_{total} &= n_i \cdot n_j \\
  &= (b-a+1) \cdot (d-c+1)
\end{align*}
\\
The inner loop will run for $(b-a+1)(d-c+1)$ iterations.


\section*{Exercise Set 9.3}
\subsection*{5 - a}
$9\cdot10\cdot10\cdot10\cdot2 = 18000$
\subsection*{8 - c}
Let $N_{p,q}$ represent the number of license plates with $p$ inital digits and $q$ alphabetical characters. 
As an example, $N_{0,4}$ is the number of license plates with 0 initial digits and 4 alphabetical characters.
\\ \\
The total number of possible license plates is therefore:
\begin{align*}
  N_{total} &= N_{0,4} + N_{1,4} + N_{0,5} + N_{1,5} \\
  &= (26^4) + (10\cdot26^4) + (26^5) + (10\cdot26^5) \\
  &= 135721872
\end{align*}
\\ 
Let $U_{p,q}$ represent the number of unique license plates, where $p$ and $q$ retain their prior definition.
\\ \\
The total number of unique license plates is therefore:
\begin{align*}
  U_{total} &= U_{0,4} + U_{1,4} + U_{0,5} + U_{1,5} \\
  &= (26\cdot25\cdot24\cdot23) + (10\cdot26\cdot25\cdot24\cdot23) + (26\cdot25\cdot24\cdot23\cdot22) + (10\cdot26\cdot25\cdot24\cdot23\cdot22) \\
  &= 358800 + 3588000 + 7893600 + 78936000 \\
  &= 90776400
\end{align*}
\\
The number of license plates that have at least one repeated letter would be the difference of the total number of license plates and the total number of unique license plates:
\begin{align*}
  N_{total} - U_{total} &= 135721872 - 90776400 = 44945472
\end{align*}
\\
\underline{Answer}: There are 44945472 license plates that have at least one repeated letter.
\subsection*{29 - f}
Looking at the the first 8-bit ID, converting from base-2 to base-10:
\begin{align*}
  11000000_2 &= (2^7+ 2^6)_{10} = 192_{10} \\
  11011111_2 &= (255 - 2^5)_{10} = 223_{10} \\
\end{align*}
\\ 
Hence, the first 8-bit ID ranges from 192 to 223 in decimal format. The other IDs range from 0 to 255 since there are no restrictions on these values.
\\ \\
The dotted decimal form an IP address for a computer in a Class C network can be represented as $w.x.y.z$, where:
\begin{align*}
  192 \leq &w \leq 223 \\
  0 \leq &x \leq 255 \\
  0 \leq &y \leq 255 \\
  0 \leq &z \leq 255
\end{align*}

\subsection*{29 - g}
The host ID for a Class C network is represented by only one 8-bit ID. The only other limitation is that it may not consist of all 0's or 1's. Hence:
\begin{align*}
  N_{host, C} &= 255 - 2 \\
  &= 253
\end{align*}
There are 253 different host IDs in a Class C network.
\subsection*{33 - e}
\begin{align*}
  N(2\cap3) - N(1 \cap 2 \cap 3) &= 3 - 2 \\
  &= 1
\end{align*}

\subsection*{33 - f}
\begin{align*}
  &N(2) - [N(1\cap2) - N(1\cap2\cap3)] - N(1\cap2\cap3) -[N(2\cap3) - N(1\cap2\cap3)] \\
  &= 26 - (8-2) - (2) - (3-2) \\
  &= 17
\end{align*}

\subsection*{34 - d}
Let's start by solving for $N(A \cap B \cap C)$:
\begin{align*}
  N(A \cup B\cup C) &= N(A) + N(B) + N(C) - N(A\cap B) - N(A\cap C) - N(B\cap C) + N(A\cap B\cap C) \\
  41 &= 21+21+31-9-14 + N(A\cap B\cap C) \\
  6 &= N(A\cap B\cap C) \\
\end{align*}
\\
Now we can solve for the case of A \textit{only}, respresented as $N_A$:
\begin{align*}
  N_A &= N(A) - N(A \cap B) - N(A \cap C) + N(A \cap B \cap C) \\
  &= 21 - 9 - 14 + 6 \\
  &= 4 \\
\end{align*}
\\ 
\underline{Answer}: There are only 4 subjects who got relief from drug A.
\newpage
\section*{Canvas Problem}
Let $N(x)$ represent the number of integers between 0 and 1000 that are multiples of $x$. 
For example, $N(6)$ is the number integers between 0 and 1000 that are multiples of 6.
\\ \\
$N(x)$ can be calculated by floor dividing 1000 by $x$. Mathematically, $N(x) = \lfloor 1000 / x \rfloor$.
\\
Next, let's calculate the number of integers that are multiples of 6, 9 or both 
\begin{align*}
  N(6) &= \lfloor 1000 / 6 \rfloor \\
  &= 166 \\
  \ \\
  N(9) &= \lfloor 1000 / 9 \rfloor \\
  &= 111 \\
  \ \\
  N(6 \cup 9) &= N(6) + N(9) - N(6 \cap 9)
\end{align*}
\\
By defintion of Least Common Factor (LCM), $N(6 \cap 9) = N(18)$ since 18 is the LCM of 6 and 9.
\\
\begin{align*}
  N(6 \cup 9) &= N(6) + N(9) - N(6 \cap 9) \\
  &= N(6) + N(9) - N(18) \\
  &= 166 + 111 - \lfloor 1000 / 18 \rfloor \\
  &= 166 + 111 - 55 \\
  &= 222
\end{align*}
\underline{Answer}: There are 222 integers between 0  and 1000 that are multiples of 6 or multiples of 9.
\end{document}
